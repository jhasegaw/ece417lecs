\documentclass{beamer}
\usepackage{tikz,amsmath,hyperref,graphicx,stackrel}
\usetikzlibrary{positioning,shadows,arrows,shapes,calc,dsp,chains}
\newcommand{\argmax}{\operatornamewithlimits{argmax}}
\newcommand{\argmin}{\operatornamewithlimits{argmin}}
\mode<presentation>{\usetheme{Frankfurt}}
\AtBeginSection[]
{
  \begin{frame}<beamer>
    \frametitle{Outline}
    \tableofcontents[currentsection,currentsubsection]
  \end{frame}
}
\title{Lecture 5: Multidimensional Signal Processing}
\author{Mark Hasegawa-Johnson}
\date{ECE 417: Multimedia Signal Processing, Fall 2021}  
\begin{document}

% Title
\begin{frame}
  \maketitle
\end{frame}

\begin{frame}
Reading: \href{https://www.google.com/books/edition/Multidimensional_Signal_Image_and_Video/0lJ0atc5X-UC?hl=en&gbpv=1&dq=multidimensional+signal+processing+tutorial&pg=PP1&printsec=frontcover}{\color{blue}Multidimensional Signal, Image, and Video Processing and Coding}, John Woods, Chapter 1
\end{frame}

% Title
\begin{frame}
  \tableofcontents
\end{frame}


%%%%%%%%%%%%%%%%%%%%%%%%%%%%%%%%%%%%%%%%%%%%
\section[Signals]{Multidimensional Signals}
\setcounter{subsection}{1}

\begin{frame}
  \frametitle{What is a Multidimensional Signal?}

  A multidimensional signal is one that can be indexed in many
  directions.  For example, a typical video that you would play on
  your laptop is a 4-dimensional signal, $x[k,t,r,c]$:
  \begin{itemize}
  \item $k$ indexes color ($k=0$ for red, $k=1$ for blue, $k=2$ for green)
  \item $t$ is the frame index
  \item $r$ is the row index
  \item $c$ is the column index
  \end{itemize}
  If there are 3 colors, 30 frames/second, 480 rows and 640 columns,
  with one byte per pixel, then that's $3\times 30\times 480\times
  640=27684000$ bytes/sec.
\end{frame}
  
\begin{frame}
  \frametitle{Generic Indexing of Multidimensional Signals}

  When we don't care about the meaning of the indices, we'll often
  talk about $x[n_1,n_2]$, where
  \begin{itemize}
  \item $n_1$ is the index along the first dimension
  \item $n_2$ is the index along the second dimension
  \item Anything we say about two dimensions can usually be extended
    to 3 or 4 dimensions, unless specified otherwise
  \end{itemize}
\end{frame}

\begin{frame}
  \frametitle{Vector Indexing of Multidimensional Signals}

  For convenience, we'll sometimes use a vector index.
  \[
  x[\vec{n}] = x[n_1,n_2],~~~\vec{n}=\left[\begin{array}{c}n_1\\n_2\end{array}\right]
  \]
  The use of vector indices is a way of abstracting away from the
  exact dimension of the problem.  In general, $\vec{n}$ might be 2d,
  3d, 4d, or whatever dimension is necessary for the problem at hand.
\end{frame}

\begin{frame}
  \frametitle{2D Delta Function}
  \begin{columns}
    \begin{column}{0.5\textwidth}
      \begin{block}{}
        Example: the two-dimensional delta function is
        \[
        \delta[n_1,n_2] = \begin{cases}
          1 & n_1=0,n_2=0\\
          0 & \mbox{otherwise}
        \end{cases}
        \]
      \end{block}
    \end{column}
    \begin{column}{0.5\textwidth}
      \begin{block}{}
        \begin{center}
          \includegraphics[width=\textwidth]{exp/delta.png}
        \end{center}
      \end{block}
      \end{column}
  \end{columns}
\end{frame}

\begin{frame}
  Example: this signal, $f[n_1,n_2]$, is an image of Joseph Fourier,
  simplified from a public domain image available on
  \href{https://en.wikipedia.org/wiki/Joseph_Fourier}{\color{blue}Wikipedia}.
  Here, $n_1$ is the row index, $n_2$ is the column index.

  Notice that, as in most images, white is the highest possible
  amplitude.  Thus the background regions (in the image on the left)
  show up as very high amplitude regions (in the 3D mesh plot on the
  right).
  
  \centerline{\includegraphics[height=0.5\textheight]{exp/fourier.jpg}\includegraphics[height=0.5\textheight]{exp/fourier_2dmesh.png}}
\end{frame}

%%%%%%%%%%%%%%%%%%%%%%%%%%%%%%%%%%%%%%%%%%%%
\section[Fourier]{Fourier Transform}
\setcounter{subsection}{1}

\begin{frame}
  \frametitle{Multidimensional Fourier Transform}

  The 2d Fourier transform is
  \begin{displaymath}
    X(\omega_1,\omega_2) = \sum_{n_1}\sum_{n_2}
    x[n_1,n_2]e^{-j\left(\omega_1n_1+\omega_2n_2\right)}
  \end{displaymath}
  Vector indexing of the image can be matched by vector indexing of
  the frequency domain, e.g.,
  $\vec\omega=[\omega_1,\omega_2]^T$.  In that case,
  \begin{displaymath}
    X(\vec\omega) = \sum_{\vec{n}}
    x[\vec{n}]e^{-j\vec\omega^T\vec{n}}
  \end{displaymath}
\end{frame}

\begin{frame}
  \frametitle{Example: Fourier Transformed}
  
  For example, here is the magnitude $|F(\omega_1,\omega_2)|$ of the
  2D Fourier transform of the image of Joseph Fourier.  Notice that,
  like most images, it has almost all of its energy at very low
  frequencies (near $\omega_1\approx 0, \omega_2\approx 0$).

  \centerline{\includegraphics[width=0.5\textwidth]{exp/fourier_transformed_image.png}\includegraphics[width=0.5\textwidth]{exp/fourier_transformed.png}}
\end{frame}


%%%%%%%%%%%%%%%%%%%%%%%%%%%%%%%%%%%%%%%%%%%%
\section[Systems]{Multidimensional Systems}
\setcounter{subsection}{1}


\begin{frame}
  \frametitle{Multidimensional Systems}
  
  A multidimensional system $y[\vec{n}]=T\left\{x[\vec{n}]\right\}$
  takes in a signal $x[\vec{n}]$, and outputs a signal $y[\vec{n}]$.
  
  \centerline{
    \begin{tikzpicture}
      \node[dspnodeopen,dsp/label=left] (x) at (0,0) {$x[\vec{n}]$};
      \node[dspsquare] (T) at (2,0) {$T\left\{\cdot\right\}$} edge[dspline](x);
      \node[dspnodeopen,dsp/label=right] (y) at (4,0) {$y[\vec{n}]$} edge[dspline](T);
  \end{tikzpicture}}
\end{frame}

\begin{frame}
  \frametitle{Linear Systems}

  A system is {\bf linear} iff, when you add two signals at the input,
  the corresponding output is the sum of their two component outputs.  Thus, if:
  \begin{align*}
    T\left\{x_1[\vec{n}]\right\} & = y_1[\vec{n}]\\
    T\left\{x_2[\vec{n}]\right\} & = y_2[\vec{n}]
  \end{align*}
  then $T\left\{\cdot\right\}$ is linear if and only if:
  \begin{align*}
    T\left\{x_1[\vec{n}]+x_2[\vec{n}]\right\} &= y_1[\vec{n}]+y_2[\vec{n}]
  \end{align*}
  As a special case of the above, scaling the input by any constant
  $a$ causes the output to scale by the same constant, i.e.,
  $T\left\{a x[\vec{n}]\right\} = ay[\vec{n}]$.
\end{frame}

\begin{frame}
  \frametitle{Shift-Invariant Systems}

  A system is {\bf shift-invariant} iff, when you shift the input
  signal by any offset, the corresponding output is shifted by the
  same offset vector.  Thus if
  \begin{displaymath}
    T\left\{x[n_1,n_2]\right\} = y[n_1,n_2]
  \end{displaymath}
  then $T\left\{\cdot\right\}$ is shift-invariant if and only if:
  \begin{displaymath}
    T\left\{x[n_1-m_1,n_2-m_2]\right\} = y[n_1-m_1,n_2-m_2]
  \end{displaymath}
\end{frame}


%%%%%%%%%%%%%%%%%%%%%%%%%%%%%%%%%%%%%%%%%%%%
\section[Convolution]{Convolution}
\setcounter{subsection}{1}

\begin{frame}
  \frametitle{Multidimensional Convolution}

  Any linear, shift-invariant system can be implemented as a
  convolution.  2D convolution is defined as
  \begin{align*}
    y[n_1,n_2] &= x[n_1,n_2]\ast h[n_1,n_2]\\
    &= \sum_{m_1=-\infty}^\infty\sum_{m_2=-\infty}^\infty x[m_1,m_2]h[n_1-m_1,n_2-m_2]
  \end{align*}
  We can generalize to vector indices like this:  
  \begin{align*}
    y[\vec{n}] &= x[\vec{n}]\ast h[\vec{n}]\\
    &=\sum_{\vec{m}} x[\vec{m}]h[\vec{n}-\vec{m}]
  \end{align*}
  Notice that this is expensive!  If both $x[\vec{n}]$ and
  $h[\vec{n}]$ are $100\times 100$ signals, then the convolution
  requires $(100)^4$ multiplications.
\end{frame}

\begin{frame}
  \frametitle{Impulse Response}

  The function $h[n_1,n_2]$ has many names.  We sometimes call it the
  {\bf filter} or {\bf kernel}.  We also call it the {\bf impulse
    response} or {\bf point spread function} (PSF) because it is the
  response of the system to an input impulse:

  \begin{align*}
    \delta[n_1,n_2]\ast h[n_1,n_2] &=
    \sum_{m_1=-\infty}^\infty\sum_{m_2=-\infty}^\infty \delta[m_1,m_2]h[n_1-m_1,n_2-m_2] \\
    & = h[n_1,n_2]
  \end{align*}
\end{frame}

\begin{frame}
  \frametitle{Example: Gaussian Blur}
  For example, consider the impulse response shown below.  This
  is a Gaussian blur kernel of size $(2M+1)\times(2M+1)$, with variance
  $\sigma^2=1.5$, meaning that
  \begin{displaymath}
    h[n_1,n_2] = \begin{cases}
      \frac{1}{2\pi\sigma^2}e^{-\left(\frac{n_1^2+n_2^2}{2\sigma^2}\right)} & -M\le n_1,n_2\le M\\
      0 & \mbox{otherwise}
    \end{cases}
  \end{displaymath}
  \centerline{\includegraphics[height=0.5\textheight]{exp/gauss_2d.png}}
\end{frame}

\begin{frame}
  \frametitle{Example: Gaussian Blur}

  Filtering an image through any lowpass filter results in a smoothed
  version of the same image.  Here's what we get when we convolve the
  image of Fourier with the Gaussian blur filter:

  \centerline{\includegraphics[height=0.45\textheight]{exp/fourier.jpg}$\rightarrow$\includegraphics[height=0.55\textheight]{exp/fourier_smoothed.png}}
\end{frame}

\begin{frame}
  \frametitle{Frequency Response}

  The frequency response of an LSI system is the Fourier transform of
  its impulse response:
  \[
  H(\vec\omega) = \sum_{\vec{n}} h[\vec{n}] e^{-j\vec\omega^T\vec{n}}
  \]
  The Fourier transform of convoluton is multiplication:
  \[
  y[\vec{n}]=x[\vec{n}]\ast h[\vec{n}] \Leftrightarrow Y(\vec\omega)=H(\vec\omega)X(\vec\omega)
  \]
\end{frame}

\begin{frame}
  \frametitle{Example: Gaussian Blur}

  A Gaussian blur kernel is a lowpass-filter, meaning that it has
  higher energy at low frequencies than at high frequencies.  Here is
  the Gaussian kernel, and its Fourier transform.

  \centerline{\includegraphics[height=0.6\textheight]{exp/gauss_2d.png}$\Leftrightarrow$
    \includegraphics[height=0.6\textheight]{exp/gauss_transformed.png}}
\end{frame}


%%%%%%%%%%%%%%%%%%%%%%%%%%%%%%%%%%%%%%%%%%%%%%%%%%%%%%%%%%%%%%%%%%%%%%%%%%%%%%%%%%%%%
\section{Separable Filtering}
\setcounter{subsection}{1}

\begin{frame}
  \frametitle{Computational Complexity of Regular Convolution}

  Recall the formula for regular convolution:
  \begin{align*}
    y[n_1,n_2] &= x[n_1,n_2]\ast h[n_1,n_2]\\
    &= \sum_{m_1=-\infty}^\infty\sum_{m_2=-\infty}^\infty x[m_1,m_2]h[n_1-m_1,n_2-m_2]
  \end{align*}
  This is an extremely expensive operation.  If $h[\vec{n}]$ and
  $x[\vec{n}]$ are both $N\times N$ signals, then convolution requires
  $N^4$ multiplications.
\end{frame}

\begin{frame}
  \frametitle{Reduced Computation}

  There are two ways computation can be reduced:
  \begin{itemize}
    \item Keep the kernel $h[\vec{n}]$ very small.  This is the method
      usually used in neural nets, e.g., a $3\times 3$ kernel is common.
    \item Use separable convolution.  This is the method usually used
      for filters that are designed by hand.
  \end{itemize}
\end{frame}

\begin{frame}
  \frametitle{Separable Filters}
  A filter $h[n_1,n_2]$ is called ``separable'' if it can be written as
  \[
  h[n_1,n_2] = h_1[n_1]h_2[n_2]
  \]
  Not all filters are separable.  For example, the diamond window
  shown below is not separable.
  \centerline{\includegraphics[height=0.5\textheight]{exp/diamond_image.png}\includegraphics[height=0.5\textheight]{exp/diamond.png}}
\end{frame}

\begin{frame}
  \frametitle{Example: Gaussian Blur}

  An important example of a separable filter is the Gaussian blur
  filter, with variance $\sigma^2$:
  \begin{align*}
    h[n_1,n_2] &= \frac{1}{2\pi\sigma^2}e^{-\left(\frac{n_1^2+n_2^2}{2\sigma^2}\right)},~~~ -M\le n_1,n_2\le M\\
    &= \left(\frac{1}{\sqrt{2\pi\sigma^2}}e^{-\frac{n_1^2}{2\sigma^2}}\right)
    \left(\frac{1}{\sqrt{2\pi\sigma^2}}e^{-\frac{n_2^2}{2\sigma^2}}\right),~~~-M\le n_1,n_2\le M\\
    &= h_1[n_1]h_2[n_2]
  \end{align*}
\end{frame}
  
\begin{frame}
  \frametitle{Example: Gaussian Blur}
  \begin{columns}
  \begin{column}{0.5\textwidth}
    This 2D filter:
    \begin{center}
      \includegraphics[width=\textwidth]{exp/gauss_2d.png}
    \end{center}
  \end{column}
  \begin{column}{0.5\textwidth}
    \ldots is the product of two 1D filters, each with this form:
    \begin{center}
      \includegraphics[width=\textwidth]{exp/gauss_1d.png}
    \end{center}
  \end{column}
  \end{columns}
\end{frame}


\begin{frame}
  \frametitle{Separable Convolution}

  If a filter is separable, then the computational cost of convolution can be reduced
  by the following trick:
  
  \begin{align*}
    y[n_1,n_2] &= x[n_1,n_2]\ast h[n_1,n_2]\\
    &= \sum_{m_1=-\infty}^\infty\sum_{m_2=-\infty}^\infty h[m_1,m_2]x[n_1-m_1,n_2-m_2]\\
    &= \sum_{m_1=-\infty}^\infty h_1[m_1]\left(\sum_{m_2=-\infty}^\infty h_2[m_2]x[n_1-m_1,n_2-m_2]\right)\\
  \end{align*}
\end{frame}

\begin{frame}
  \frametitle{Separable Convolution: Computational Complexity}
  
  \begin{align*}
    x[\vec{n}]\ast h[\vec{n}] &=
    \sum_{m_1=-\infty}^\infty h_1[m_1]\left(\sum_{m_2=-\infty}^\infty h_2[m_2]x[n_1-m_1,n_2-m_2]\right)\\
  \end{align*}
  
  \begin{itemize}
    \item The part inside the parentheses computes a 1D convolution,
      which is an $N^2$ operation, for each of the $N$ rows.  Cost:
      $N^3$.
    \item The part outside the parentheses then computes a 1D
      convolution for each of the $N$ columns.  Cost: $N^3$.
    \item Total cost: $2N^3$ computations.
  \end{itemize}
  If $N$ is large (e.g., 1000), then often, $2N^3$ is a quite
  reasonable number (e.g., two billion computations), while $N^4$ is
  quite unreasonable (e.g., a trillion).
\end{frame}

\begin{frame}
  \frametitle{Separable Convolution: Notation}

  It is sometimes useful to have a special notation for separable
  convolution.  For example, we could define the {\bf row convolution}
  operator $\ast_2$, and the {\bf column convolution} operator $\ast_1$:
  \begin{align*}
    h_2[n_2] \ast_2 x[n_1,n_2] &\equiv \sum_{m_2=-\infty}^\infty h_2[m_2]x[n_1, n_2-m_2]\\
    h_1[n_1] \ast_1 x[n_1,n_2] &\equiv \sum_{m_1=-\infty}^\infty h_1[m_1]x[n_1-m_1, n_2]
  \end{align*}
  Then:
  \begin{align*}
    x[n_1,n_2]\ast h[n_1,n_2]
    &= h_1[n_1] \ast_1 h_2[n_2]\ast_2 x[n_1,n_2]
  \end{align*}
  
\end{frame}

\begin{frame}
  \frametitle{Separable Convolution: How to Do It}

  Using \texttt{numpy}, you will almost always want to use separable
  convolutions.  To do that, you need to define some variable
  $v[\vec{n}]$ which is intermediate between $x[\vec{n}]$ and
  $y[\vec{n}]$.  Then you do the following:
  \begin{itemize}
  \item Compute $v[\vec{n}]=h_2[n_2]\ast_2 x[\vec{n}]$, i.e., convolve
    $h_2$ with each of the rows of the image.
  \item Compute $y[\vec{n}]=h_1[n_1]\ast_1 v[\vec{n}]$, i.e., convolve
    $h_1$ with each of the columns of the image.
  \end{itemize}
  In fact, that's how I computed the Gaussian-blurred image of Fourier.
\end{frame}


\begin{frame}
  \frametitle{Other facts about separable filters}

  \begin{itemize}
  \item The Fourier transform is a separable operation, so it can be
    done most efficiently by first transforming each row, then
    transforming each column.  This is what \texttt{np.fft.fft2} does for
    you.
    \[
    X(\omega_1,\omega_2) = \sum_{n_1} e^{-j\omega_1 n_1}\left(\sum_{n_2} e^{-j\omega_2 n_2} x[n_1,n_2]\right)
    \]
  \item If a filter is separable, then its Fourier transform is also separable:
    \[
    h[n_1,n_2]=h_1[n_1]h_2[n_2] \Leftrightarrow
    H(\omega_1,\omega_2)=H_1(\omega_1)H_2(\omega_2)
    \]
  \end{itemize}
\end{frame}


%%%%%%%%%%%%%%%%%%%%%%%%%%%%%%%%%%%%%%%%%%%%%%%%%%%%%%%%%%%%%%%%%%%%%%%%%%%%%%%%%%%%%
\section{Examples}
\setcounter{subsection}{1}

\begin{frame}
  \frametitle{Pencil and Paper Examples}

  Consider the $M\times M$ rectangular window signal:
  \begin{displaymath}
    h[n_1,n_2] = \begin{cases}
      1 & 0\le n_1,n_2 \le M-1\\
      0 & \mbox{otherwise}
    \end{cases}
  \end{displaymath}
  Is this a separable filter?  Is it a lowpass or a highpass filter?
  Write $x[\vec{n}]\ast h[\vec{n}]$ --- what is another name for the
  output of this filter?
  What is the frequency response $H(\vec\omega)$?
\end{frame}

\begin{frame}
  \frametitle{Jupyter Examples}

  Create a rectangular window with this form:
  \begin{displaymath}
    h[n_1,n_2] = \begin{cases}
      1 & 0\le n_1,n_2 \le M-1\\
      0 & \mbox{otherwise}
    \end{cases}
  \end{displaymath}
  Convolve it with some image, and show the result.  Take its Fourier
  transform using \texttt{np.fft.fft2} (making sure to zero-pad it to
  a large $N$, maybe $N=7M$ or so), then fftshift it using
  \texttt{np.fft.fftshift}, and show the absolute value of the result
  using \texttt{matplotlib.pyplot.plot\_wireframe}.
\end{frame}

%%%%%%%%%%%%%%%%%%%%%%%%%%%%%%%%%%%%%%%%%%%%%%%%%%%%%%%%%%%%%%%%%%%%%%%%%%%%%%%%%%%%%
\section{Summary}
\setcounter{subsection}{1}

\begin{frame}
  \frametitle{Summary}
  \begin{itemize}
  \item {\bf Fourier Transform:}
    \[
    H(\vec\omega) = \sum_{\vec{n}} h[\vec{n}] e^{-j\vec\omega^T\vec{n}}
    \]
  \item {\bf Convolution:}
    \[
    h[\vec{n}]\ast x[\vec{n}] = \sum_{\vec{m}} h[\vec{m}] x[\vec{n}-\vec{m}]
    \]
  \item {\bf Separable Filtering:}
    If $h[n_1,n_2]=h_1[n_1]h_2[n_2]$, then
    \[
    h[\vec{n}]\ast x[\vec{n}] = h_1[n_1] \ast_1 \left(h_2[n_2]\ast_2 x[\vec{n}]\right)
    \]
  \end{itemize}
\end{frame}


\end{document}

