\documentclass{beamer}
\usepackage{tikz,amsmath,hyperref,graphicx,stackrel,animate,media9}
\usetikzlibrary{positioning,shadows,arrows,shapes,calc}
\newcommand{\argmax}{\operatornamewithlimits{argmax}}
\newcommand{\argmin}{\operatornamewithlimits{argmin}}
\mode<presentation>{\usetheme{Frankfurt}}
\AtBeginSection[]
{
  \begin{frame}<beamer>
    \frametitle{Outline}
    \tableofcontents[currentsection,currentsubsection]
  \end{frame}
}
\title{Lecture 1: Intro; DSP Review}
\author{Mark Hasegawa-Johnson}
\date{ECE 417: Signal and Image Analysis, Fall 2021}  
\begin{document}

% Title
\begin{frame}
  \maketitle
\end{frame}

% Title
\begin{frame}
  \tableofcontents
\end{frame}

%%%%%%%%%%%%%%%%%%%%%%%%%%%%%%%%%%%%%%%%%%%%
\section[Intro]{Intro: Multimedia  Signal Processing}
\setcounter{subsection}{1}

\begin{frame}
  \frametitle{Intro: What is Signal Processing?}

  \begin{itemize}
  \item Linear Time Invariant systems: Use Fourier analysis
    \begin{itemize}
      \item Advantages: analytic solutions, instantaneously, no training
      \item Disadvantages: only optimal for LTI systems
    \end{itemize}
  \item Nonlinear \& Time-Varying systems: Use machine learning
    \begin{itemize}
    \item Advantages: learns the optimal solution for any problem
    \item Disadvantages: training takes time, often fails; analytic solutions provide only loose
      upper bounds
    \end{itemize}
  \end{itemize}

\end{frame}

\begin{frame}
  \frametitle{Where can I learn more?}

  There are many excellent conferences and publications  related to particular
  areas in signal processing.  There is also one really excellent conference
  and one really excellent magazine dedicated specifically to signal processing:
  \begin{itemize}
  \item \href{https://2022.ieeeicassp.org/}{ICASSP} (International
    Conference on Acoustics, Speech and Signal Processing)
  \item \href{https://signalprocessingsociety.org/publications-resources/ieee-signal-processing-magazine}{IEEE Signal Processing Magazine}
  \end{itemize}
\end{frame}

%%%%%%%%%%%%%%%%%%%%%%%%%%%%%%%%%%%%%%%%%%%%
\section[Overview]{Overview: Course Policies and Administration}
\setcounter{subsection}{1}

\begin{frame}
  \frametitle{Overview: Course Topics}

  There are six MPs, covering
  $\{\mbox{speech},\mbox{video}\}\times\{\mbox{DSP
    methods},\mbox{Bayesian methods},\mbox{ML methods}\}$:
  \begin{itemize}
  \item DSP methods: MP1 (speech synthesis), MP2 (video synthesis)
  \item Bayesian methods: MP3 (image recognition), MP4 (speech recognition)
  \item Neural networks: MP5 (image recognition), MP6 (voice conversion)
  \end{itemize}
\end{frame}

\begin{frame}
  \frametitle{Overview: Course  Policies and Administration}

  About here, visit the 
  \href{https://courses.grainger.illinois.edu/ece417/fa2021/}{course webpage}.

  Notice: if you haven't had a DSP course yet, you might want to
  consider taking
  \href{https://courses.grainger.illinois.edu/ece401/fa2021/}{ECE 401}
  instead of this course.
\end{frame}

%%%%%%%%%%%%%%%%%%%%%%%%%%%%%%%%%%%%%%%%%%%%
\section[Fourier]{Review: Fourier Transforms}
\setcounter{subsection}{1}

\begin{frame}
  \frametitle{Review: Fourier Transforms}

  About here, watch the movie called
  \href{https://mediaspace.illinois.edu/media/t/1_x7tsa6l7/26816181}{fourier\_review.mov}.
\end{frame}

%%%%%%%%%%%%%%%%%%%%%%%%%%%%%%%%%%%%%%%%%%%%
\section[Poles/Zeros]{Review: Poles and Zeros}
\setcounter{subsection}{1}

\begin{frame}
  \frametitle{Z Transform}
  You might remember that the Z-transform is, basically, a funny way to write the
  DTFT.  The DTFT is
  \[
  H(\omega) = \sum_{n=-\infty}^\infty h[n]e^{-j\omega n}
  \]
  \ldots and the Z  transform is
  \[
  H(z) = \sum_{n=-\infty}^\infty h[n]z^{-n}
  \]
  There are some limits on the values of $z$ for which $H(z)$ is
  finite.  Let's talk about those.
\end{frame}

\begin{frame}
  \frametitle{FIR and IIR Filters}

  \begin{itemize}
  \item An {\bf FIR} (finite impulse response) filter is one
    whose impulse response lasts a finite amount of time.  Any such filter
    can be written as
    \[
    y[n] = \sum_{m=0}^{M-1} b_m x[n-m]
    \]
  \item An {\bf IIR} (infinite impulse response) filter is one whose
    impulse response lasts an infinite amount of time.  We will be
    most interested in IIR filters that can be written as
    \[
    \sum_{m=0}^{N-1} a_n y[n-m] = \sum_{m=0}^{M-1} b_m x[n-m]
    \]
  \end{itemize}
  
\end{frame}


\begin{frame}
  \frametitle{First-Order Feedback-Only Filter}

  Let's find the general form of $h[n]$, for the simplest possible
  IIR filter: a filter with one feedback term, and no
  feedforward terms, like this:
  \[
  y[n] = x[n] + ay[n-1],
  \]
  where $a$ is any constant (positive, negative, real, or complex).
\end{frame}

\begin{frame}
  \frametitle{Impulse Response of a First-Order Filter}

  We can find the impulse response by putting in $x[n]=\delta[n]$, and
  getting out $y[n]=h[n]$:
  \[
  h[n] = \delta[n] + ah[n-1].
  \]
  Recursive computation gives
  \begin{align*}
    h[0] &= 1 \\
    h[1] &= a\\
    h[2] &= a^2\\
     & \vdots\\
    h[n] &= a^nu[n]
  \end{align*}
  where we use the notation $u[n]$ to mean the ``unit step function,''
  \[u[n] = \begin{cases}1& n\ge 0\\0 & n<0\end{cases}\]
\end{frame}

\begin{frame}
  \frametitle{Impulse Response of Stable First-Order Filters}

  The coefficient, $a$, can be positive, negative, or even complex.
  If $a$ is complex, then $h[n]$ is also complex-valued.
  \centerline{\includegraphics[height=2.5in]{exp/iir_stable.png}}

\end{frame}

\begin{frame}
  \frametitle{Impulse Response of Unstable First-Order Filters}

  If $|a|>1$, then the impulse response grows exponentially.  If
  $|a|=1$, then the impulse response never dies away.  In either case,
  we say the filter is ``unstable.''
  \centerline{\includegraphics[height=2.5in]{exp/iir_unstable.png}}

\end{frame}

\begin{frame}
  \frametitle{Instability}

  \begin{itemize}
  \item A {\bf stable} filter is one that always generates finite
    outputs ($|y[n]|$ finite) for every possible finite input
    ($|x[n]|$ finite).
  \item An {\bf unstable} filter is one that, at least sometimes,
    generates infinite outputs, even if the input is finite.
  \item A first-order IIR filter is stable if and only if $|a|<1$.
  \end{itemize}
\end{frame}


\begin{frame}
  \frametitle{Transfer Function of a First-Order Filter}

  We can find the transfer function by taking the Z-transform of each
  term in this equation:
  \[
  y[n] = x[n] + ay[n-1].
  \]
  Using the transform pair $y[n-1]\leftrightarrow z^{-1}Y(z)$, we get
  \[
  Y(z) = X(z)+az^{-1} Y(z),
  \]
  which we can solve to get
  \[
  H(z)  = \frac{Y(z)}{X(z)} = \frac{1}{1-az^{-1}}
  \]
\end{frame}

\begin{frame}
  \frametitle{Frequency Response of a  First-Order Filter}

  If the filter is stable ($|a|<1$), then 
  we can find the frequency response by plugging in $z=e^{j\omega}$:
  \[
  H(\omega) = H(z)\vert_{z=e^{j\omega}}  =  \frac{1}{1-ae^{-j\omega}}~~~\mbox{if}~|a|<1
  \]

  This formula only works if $|a|<1$.
\end{frame}

\begin{frame}
  \frametitle{Frequency Response of a  First-Order Filter}
  \[
  H(\omega) = \frac{1}{1-ae^{-j\omega}}~~~\mbox{if}~|a|<1
  \]
  
  \centerline{\includegraphics[width=4.5in]{exp/iir_freqresponse.png}}
\end{frame}

\begin{frame}
  \frametitle{Transfer Function $\leftrightarrow$ Impulse Response}

  Notice that $H(z)$ is actually the $Z$-transform of $h[n]$.  We can
  prove that as follows:
  \begin{align*}
    H(z) &= \sum_{n=-\infty}^\infty h[n] z^{-n}\\
    &= \sum_{n=0}^\infty a^n z^{-n} 
  \end{align*}
  This is a standard geometric series, with a ratio of $az^{-1}$.  As
  long as $|a|<1$, we can use the formula for an infinite-length
  geometric series, which is:
  \[
  H(z) = \frac{1}{1-az^{-1}},
  \]
  So we confirm that $h[n]\leftrightarrow H(z)$ for both FIR and IIR
  filters, as long as $|a|<1$.
\end{frame}

\begin{frame}
  \frametitle{First-Order Filter}

  Now, let's find the transfer function of a general first-order filter, including BOTH
  feedforward and feedback delays:
  \[
  y[n] = x[n] + bx[n-1] + ay[n-1],
  \]
  where we'll assume that $|a|<1$, so the filter is stable.  
\end{frame}

\begin{frame}
  \frametitle{Transfer Function of a First-Order Filter}

  We can find the transfer function by taking the Z-transform of each
  term in this equation equation:
  \begin{align*}
    y[n] &= x[n] + bx[n-1] + ay[n-1],\\
    Y(z) &= X(z)+bz^{-1}X(z)+az^{-1} Y(z),
  \end{align*}
  which we can solve to get
  \[
  H(z)  = \frac{Y(z)}{X(z)} = \frac{1+bz^{-1}}{1-az^{-1}}.
  \]
\end{frame}

\begin{frame}
  \frametitle{Treating $H(z)$ as a Ratio of Two Polynomials}

  Notice that $H(z)$ is the ratio of two polynomials:
  \[
  H(z)=\frac{1+bz^{-1}}{1-az^{-1}}=\frac{z+b}{z-a}
  \]
  \begin{itemize}
  \item $z=-b$ is called the {\bf zero} of $H(z)$, meaning that $H(-b)=0$.
  \item $z=a$ is called the {\bf pole} of $H(z)$, meaning that $H(a)=\infty$
  \end{itemize}
\end{frame}

\begin{frame}
  \frametitle{The Pole and Zero of $H(z)$}

  \begin{itemize}
  \item The pole, $z=a$, and zero, $z=-b$, are the values of $z$ for which
    $H(z)=\infty$  and $H(z)=0$, respectively.
  \item But what does that mean?  We know that for $z=e^{j\omega}$,
    $H(z)$ is just the frequency response:
    \[
    H(\omega) = H(z)\vert_{z=e^{j\omega}}
    \]
    but the pole and zero do not normally have unit magnitude.
  \item What it means is that:
    \begin{itemize}
      \item When $\omega=\angle (-b)$, then
        $|H(\omega)|$ is as close to a zero as it can possibly get, so at that 
        that frequency, $|H(\omega)|$ is as low as it can get.
      \item When $\omega=\angle a$, then
        $|H(\omega)|$ is as close to a pole as it can possibly get, so at that 
        that frequency, $|H(\omega)|$ is as high as it can get.
    \end{itemize}
  \end{itemize}
\end{frame}

\begin{frame}
  \centerline{\animategraphics[loop,controls,width=4.5in]{10}{exp/magresponse}{0}{99}}
\end{frame}

\begin{frame}
  \centerline{\animategraphics[loop,controls,width=5in]{10}{exp/toneresponse}{0}{99}}
\end{frame}

\begin{frame}
  \frametitle{Vectors in the Complex Plane}

  Suppose we write $|H(z)|$ like this:
  \[
  \vert H(z)\vert = \frac{\vert z+b\vert}{\vert z-a\vert}
  \]
  Now let's evaluate at $z=e^{j\omega}$:
  \[
  \vert H(\omega)\vert = 
  \frac{\vert e^{j\omega}+b\vert}{\vert e^{j\omega}-a\vert}
  \]
  What we've discovered is that $|H(\omega)|$ is small when the vector
  distance $|e^{j\omega}+b|$ is small, but LARGE when the vector
  distance $|e^{j\omega}-a|$ is small.
\end{frame}

\begin{frame}
  \centerline{\animategraphics[loop,controls,width=4.5in]{10}{exp/magresponse}{0}{99}}
\end{frame}

\begin{frame}
  \frametitle{Why This is Useful}

  Now we have another way of thinking about frequency response.
  \begin{itemize}
    \item Instead of just LPF, HPF, or BPF, we can design a filter to have
      zeros at particular frequencies, $\angle (-b)$, AND to have
      poles at particular frequencies, $\angle a$,
    \item The magnitude $|H(\omega)|$ is
      $|e^{j\omega}+b|/|e^{j\omega}-a|$.
    \item Using this trick, we can design filters that have much more
      subtle frequency responses than just an ideal LPF, BPF, or HPF.
  \end{itemize}
\end{frame}


%%%%%%%%%%%%%%%%%%%%%%%%%%%%%%%%%%%%%%%%%%%%
\section[Summary]{Summary}
\setcounter{subsection}{1}

\begin{frame}
  \frametitle{Summary: DSP Review}
  In summary, you should remember how to do these things:
  \begin{itemize}
  \item Fourier Series, CTFT, DTFT, and DFT.
  \item Z-transform.
  \item Z-transform and frequency-response of FIR and IIR filters.
  \end{itemize}
\end{frame}

%%%%%%%%%%%%%%%%%%%%%%%%%%%%%%%%%%%%%%%%%%%%
\section[Optional]{Other Possibly Useful Review Videos}
\setcounter{subsection}{1}

\begin{frame}
  \frametitle{Other Possibly Useful Review Videos}

  Here are some other videos that might be useful.
  \begin{itemize}
  \item This one, I think, is a review of material from ECE 310, so I
    kind of expect you to know it:
    \href{https://mediaspace.illinois.edu/media/t/1_7vbishf8/26816181}{filtering review}
  \item These three provide more details about noise, and about
    speech.  This material is optional; if it helps you understand the
    LPC material, that's great, but if not, you can ignore it.
    \begin{itemize}
      \item \href{https://mediaspace.illinois.edu/media/t/1_naudxmet/26816181}{Noise}.
      \item \href{https://mediaspace.illinois.edu/media/t/1_gm5xuh5g/26816181}{Speech}.
      \item \href{https://mediaspace.illinois.edu/media/t/1_2839d36o/26816181}{Windwed Speech}.
    \end{itemize}
  \end{itemize}
\end{frame}


\end{document}

