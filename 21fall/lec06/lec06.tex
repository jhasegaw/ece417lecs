\documentclass{beamer}
\usepackage{tikz,amsmath,hyperref,graphicx,stackrel,animate}
\usetikzlibrary{positioning,shadows,arrows,shapes,calc,dsp,chains}
\newcommand{\argmax}{\operatornamewithlimits{argmax}}
\newcommand{\argmin}{\operatornamewithlimits{argmin}}
\mode<presentation>{\usetheme{Frankfurt}}
\AtBeginSection[]
{
  \begin{frame}<beamer>
    \frametitle{Outline}
    \tableofcontents[currentsection,currentsubsection]
  \end{frame}
}
\title{Lecture 6: Optical Flow}
\author{Mark Hasegawa-Johnson}
\date{ECE 417: Multimedia Signal Processing, Fall 2021}  
\begin{document}

% Title
\begin{frame}
  \maketitle
\end{frame}

% Title
\begin{frame}
  \tableofcontents
\end{frame}


%%%%%%%%%%%%%%%%%%%%%%%%%%%%%%%%%%%%%%%%%%%%
\section[Gradient]{Image Gradient}
\setcounter{subsection}{1}

\begin{frame}
  \frametitle{Image gradient}

  The image gradient is a way of characterizing the distribution of
  light and dark pixels in an image.  Suppose the image intensity is
  $f(t,r,c)$.  The image gradient is:
  \[
  \nabla f =
  = \left[\begin{array}{c} \frac{\partial f(t,r,c)}{\partial c}\\
      \frac{\partial f(t,r,c)}{\partial r}\end{array}\right]
  \]
\end{frame}

\begin{frame}
  \frametitle{Image gradient}

  {\centering
    \includegraphics[width=\textwidth]{exp/512px-Gradient2.svg.png}

    {\tiny
      CC-BY 2.5, Gufosawa, 2021
    }
  }
\end{frame}

\begin{frame}
  \frametitle{Image gradient}

  {\centering
    \includegraphics[width=\textwidth]{exp/Intensity_image_with_gradient_images.png}

    {\tiny
      Public domain image, Njw00, 2010
    }
  }
\end{frame}


\begin{frame}
  \frametitle{How do you calculate the image gradient?}

  Basically, use one of the standard numerical estimates of a
  derivative.  For example, the central-difference operator:
  \[
  \nabla f 
  = \left[\begin{array}{c} \frac{\partial f(t,r,c)}{\partial c}\\
      \frac{\partial f(t,r,c)}{\partial r}\end{array}\right]
  = \left[\begin{array}{c} \frac{f[t,r,c+1]-f[t,r,c-1]}{2}\\
      \frac{f[t,r+1,c]-f[t,r-1,c]}{2}\end{array}\right]
  \]

  \href{https://en.wikipedia.org/wiki/Image_derivatives}{\color{blue}Wikipedia}
  has a good listing of other methods you can use.
\end{frame}

%%%%%%%%%%%%%%%%%%%%%%%%%%%%%%%%%%%%%%%%%%%%
\section[Flow]{Optical Flow}
\setcounter{subsection}{1}

\begin{frame}
  \frametitle{Optical Flow}

  Definition: {\bf optical flow} is the vector field $\vec{v}(t,r,c)$
  specifying the current apparent velocity of the pixel at position
  $(r,c)$.  It depends on motion of (1) the object observed, and (2)
  the observer.
  
  {\centering
    \includegraphics[width=\textwidth]{exp/Opticfloweg.png}

    {\tiny
      CC-BY 2.5, Huston SJ, Krapp HG, 2008 Visuomotor Transformation
      in the Fly Gaze Stabilization System. PLoS Biol 6(7):
      e173. doi:10.1371/journal.pbio.006017
    }
  }
\end{frame}

\begin{frame}
  \frametitle{Optical Flow}

  Definition: {\bf optical flow} is the vector field $\vec{v}(t,r,c)$
  specifying the current apparent velocity of the pixel at position
  $(r,c)$.  It depends on motion of (1) the object observed, and (2)
  the observer.

  {\centering
    \includegraphics[width=\textwidth]{exp/Optical-flow-estimation-for-different-sequences.jpg}\\

    {\tiny
      Pengcheng Han et. al. "An Object Detection Method Using Wavelet
      Optical Flow and Hybrid Linear-Nonlinear Classifier",
      Mathematical Problems in Engineering doi:10.1155/2013/96541
    }
  }
\end{frame}

\begin{frame}
  \frametitle{Optical Flow}

  For example, you can use it to track a user-specified rectangle in
  the ultrasound video of a tendon.

  {\centering
    \ifdefined\novideo
    \includegraphics[width=0.7\textwidth]{exp/tendon/1.png}
    \else
    \animategraphics[loop,controls,width=0.7\textwidth]{100}{exp/tendon/}{400}{500}
    \fi

    {\tiny
      CC-BY 4.0 Chuang B, Hsu J, Kuo L, Jou I, Su F, Sun Y
      (2017). "Tendon-motion tracking in an ultrasound image sequence
      using optical-flow-based block matching". BioMedical Engineering
      OnLine
    }
  }
\end{frame}


\begin{frame}
  \frametitle{How to calculate optical flow}

  General idea:
  \begin{itemize}
    \item Treat the image as a function of continuous time and
      space, $f(t,r,c)$.
    \item If the image intensity is changing, as a function of time,
      then try to explain it by moving pixels around.
  \end{itemize}
\end{frame}

\begin{frame}
  \frametitle{Calculating optical flow}

  More formally, let's treat local variation of $f(t,r,c)$ using a
  first-order Taylor series:
  \[
  f\left(t+\Delta t,r+\Delta r,c+\Delta c\right)\approx f(t,r,c) +
  \Delta t\frac{\partial f}{\partial t}+
  \Delta r\frac{\partial f}{\partial r}+
  \Delta c\frac{\partial f}{\partial c}
  \]
  Hypothesize that all intensity variations are caused by pixels
  moving around.  Then
  \[
  f\left(t+\Delta t,r+\Delta r,c+\Delta c\right) - f(t,r,c) = 0
  \]
  
\end{frame}


\begin{frame}
  \frametitle{Calculating optical flow}

  \begin{align*}
    0 & =   f\left(t+\Delta t,r+\Delta r,c+\Delta c\right) - f(t,r,c) \\
    &\approx   \Delta t\frac{\partial f}{\partial t}+
    \Delta r\frac{\partial f}{\partial r}+
    \Delta c\frac{\partial f}{\partial c}
  \end{align*}
\end{frame}
\begin{frame}
  \frametitle{Calculating optical flow}

  \[
  0 \approx   \Delta t\frac{\partial f}{\partial t}+
  \Delta r\frac{\partial f}{\partial r}+
  \Delta c\frac{\partial f}{\partial c}
  \]
  Dividing through by $\Delta t$, and taking the limit as $\Delta
  t\rightarrow 0$, we get
  \[
  0 \approx   \frac{\partial f}{\partial t}+
  \left(\frac{\partial r}{\partial t}\right)\frac{\partial f}{\partial r}+
  \left(\frac{\partial c}{\partial t}\right)\frac{\partial f}{\partial c}
  \]
\end{frame}

\begin{frame}
  \frametitle{Calculating optical flow}

  Re-arranging gives us the optical flow equation:
  \[
  -\frac{\partial f}{\partial t} \approx 
  \left(\frac{\partial r}{\partial t}\right)\frac{\partial f}{\partial r}+
  \left(\frac{\partial c}{\partial t}\right)\frac{\partial f}{\partial c}
  \]
\end{frame}

\begin{frame}
  \frametitle{How to calculate optical flow}
  
  Define the optical flow vector, $\vec{v}(t,r,c)$, and image gradient, $\nabla f(t,r,c)$:
  \[
  \vec{v}=\left[\begin{array}{c}\frac{\partial c}{\partial t}\\\frac{\partial r}{\partial t}\end{array}\right],~~~
  \nabla f=\left[\begin{array}{c}\frac{\partial f}{\partial c}\\\frac{\partial f}{\partial r}\end{array}\right]
  \]
  Then the optical flow equation is:
  \[
  -\frac{\partial f}{\partial t} =(\nabla f)^T\vec{v}
  \]
\end{frame}

%%%%%%%%%%%%%%%%%%%%%%%%%%%%%%%%%%%%%%%%%%%%
\section[Lucas-Kanade]{The Lucas-Kanade Algorithm}
\setcounter{subsection}{1}

\begin{frame}
  \frametitle{How to calculate optical flow}

  So we have this optical flow equation:
  \[
  -\frac{\partial f}{\partial t} =(\nabla f)^T\vec{v}
  \]
  Assume that we can calculate $\partial f/\partial t$ and $\nabla f$,
  using standard image gradient methods.  Now we just need to find
  $\vec{v}$.  But $\vec{v}=[v_r, v_c]^T$ is a vector of two unknowns,
  so the equation above is one equation in two unknowns!
\end{frame}

\begin{frame}
  \frametitle{How to calculate optical flow}

  The solution is to assume that a small block of pixels all move together:
  
  {\centering
    \includegraphics[height=1.5in]{exp/Block-matching_algorithm.png}

    {\tiny CC-SA 4.0 by German iris, 2017}
  }
\end{frame}

\begin{frame}
  \frametitle{The Lucas-Kanade Algorithm}

  The Lucas-Kanade Algorithm replaces this equation
  \[
  -\frac{\partial f}{\partial t} =(\nabla f)^T\vec{v}
  \]
  with this equation:
  \[
  -\left[\begin{array}{c}
      \frac{\partial f[t,r-W,c-W]}{\partial t}\\
      \vdots\\
      \frac{\partial f[t,r+W,c+W]}{\partial t}
    \end{array}\right]
  =
  \left[\begin{array}{cc}
      \frac{\partial f[t,r-W,c-W]}{\partial c}&\frac{\partial f[t,r-W,c-W]}{\partial r}\\
      \vdots\\
      \frac{\partial f[t,r+W,c+W]}{\partial c}&\frac{\partial f[t,r+W,c+W]}{\partial r}
    \end{array}\right]
  \left[\begin{array}{c}v_c\\ v_r\end{array}\right]
  \]
  so that we are averaging over a block of size $(2W+1)\times(2W+1)$ pixels.
\end{frame}

\begin{frame}
  \frametitle{The Lucas-Kanade Algorithm}

  The Lucas-Kanade algorithm solves the equation
  \[\vec{b}=A\vec{v} \]
  where
  \[
  \vec{b} = -\left[\begin{array}{c}
      \frac{\partial f[t,r-W,c-W]}{\partial t}\\
      \vdots\\
      \frac{\partial f[t,r+W,c+W]}{\partial t}
    \end{array}\right],~~~
  A=\left[\begin{array}{cc}
      \frac{\partial f[t,r-W,c-W]}{\partial c}&\frac{\partial f[t,r-W,c-W]}{\partial r}\\
      \vdots\\
      \frac{\partial f[t,r+W,c+W]}{\partial c}&\frac{\partial f[t,r+W,c+W]}{\partial r}
    \end{array}\right]
  \]
  \[
  \vec{v}=\left[\begin{array}{c}v_c[t,r,c]\\ v_r[t,r,c]\end{array}\right]
  \]
\end{frame}


\begin{frame}
  \frametitle{The Lucas-Kanade Algorithm}

  The Lucas-Kanade algorithm solves the equation
  \[\vec{b}=A\vec{v} \]
  \ldots but now $A$ is a matrix of size $(2W+1)\times 2$, so it's
  still not invertible!  How do we solve that?
\end{frame}

%%%%%%%%%%%%%%%%%%%%%%%%%%%%%%%%%%%%%%%%%%%%
\section[Pseudo-Inverse]{Pseudo-Inverse}
\setcounter{subsection}{1}

\begin{frame}
  \frametitle{Pseudo-Inverse}

  The pseudo-inverse, $A^\dag$, of any matrix $A$, is a matrix that
  acts like $A^{-1}$ in many ways, but it doesn't require $A$ to be
  square.  Here are some of its properties:
  \begin{align*}
    A A^\dag A &= A\\
    A^\dag A A^\dag &= A
  \end{align*}
\end{frame}

\begin{frame}
  \frametitle{Pseudo-Inverse}

  Of particular interest to us, the vector $\vec{v}=A^\dag\vec{b}$
  ``pseudo-solves'' the equation $\vec{b}=A\vec{v}$.  By pseudo-solve, we
  mean that
  \begin{itemize}
  \item If $A$ is a short fat matrix, then there are an infinite
    number of different vectors $\vec{v}$ that solve
    $\vec{b}=A\vec{v}$.  $\vec{v}=A^\dag \vec{b}$ is one of those;
    specifically, it's the one that minimizes $\Vert\vec{v}\Vert^2$.
  \item If $A$ is a tall thin matrix, then there is usually no vector
    $\vec{v}$ that solves $\vec{b}=A\vec{v}$, but $\vec{v}=A^\dag
    \vec{b}$ is the vector that comes closest, in the sense that 
    \[
    A^\dag\vec{b} = \mbox{argmin}_{\vec{v}}\Vert\vec{b}-A\vec{v}\Vert^2
    \]
  \end{itemize}
\end{frame}


\begin{frame}
  \frametitle{Solving for the Pseudo-Inverse}

  Let's use this equation:
  \[
  v^* = A^\dag\vec{b} = \mbox{argmin}_v\Vert\vec{b}-A\vec{v}\Vert^2
  \]
  to solve for the pseudo-inverse.
\end{frame}

\begin{frame}
  \frametitle{Solving for the Pseudo-Inverse}
  \begin{align*}
    A^\dag\vec{b} &= \mbox{argmin}_v\Vert\vec{b}-A\vec{v}\Vert^2\\
    &= \mbox{argmin}_v\left(\vec{b}-A\vec{v}\right)^T\left(\vec{b}-A\vec{v}\right)\\
    &= \mbox{argmin}_v\left(\vec{b}^T\vec{b}-2\vec{v}^TA^T\vec{b}+\vec{v}^TA^TA\vec{v}\right)
  \end{align*}
\end{frame}

\begin{frame}
  \frametitle{Solving for the Pseudo-Inverse}
  \begin{align*}
    A^\dag\vec{b}= \mbox{argmin}_v\left(\vec{b}^T\vec{b}-2\vec{v}^TA^T\vec{b}+\vec{v}^TA^TA\vec{v}\right)
  \end{align*}
  If we differentiate the quantity in parentheses, and set the derivative to zero, we get
  \begin{align*}
    \vec{0} &= -2A^T\vec{b}+2A^TA\vec{v}
  \end{align*}
  Assume that the columns of $A$ are linearly independent; then $A^TA$
  is invertible, and so the solution is
  \[
  \vec{v} = (A^TA)^{-1}A^T\vec{b}
  \]
\end{frame}

%%%%%%%%%%%%%%%%%%%%%%%%%%%%%%%%%%%%%%%%%%%%
\section[Summary]{Summary}
\setcounter{subsection}{1}

\begin{frame}
  \frametitle{Summary: Optical Flow}

  \begin{itemize}
    \item Optical flow is the vector field, $\vec{v}(t,r,c)$, as a function of
      pixel position and frame number.
    \item It is computed by assuming that the only changes to an image
      are the ones caused by motion, so that
      \[
      f(t+\Delta t,r+\Delta r,c+\Delta c) = f(t,r,c)
      \]
    \item From that assumption, we get the optical flow equation:
      \[
      -\frac{\partial f}{\partial t} =\vec{v}^T\nabla f
      \]
  \end{itemize}
\end{frame}

\begin{frame}
  \frametitle{The Lucas-Kanade Algorithm}

  Lucas-Kanade assumes that there is a $(2W+1)\times(2W+1)$ block of
  pixels that all move together, so that $\vec{b}=A\vec{v}$, where
  \[
  \vec{b} = -\left[\begin{array}{c}
      \frac{\partial f[t,r-W,c-W]}{\partial t}\\
      \vdots\\
      \frac{\partial f[t,r+W,c+W]}{\partial t}
    \end{array}\right],~~~
  A=\left[\begin{array}{cc}
      \frac{\partial f[t,r-W,c-W]}{\partial c}&\frac{\partial f[t,r-W,c-W]}{\partial r}\\
      \vdots\\
      \frac{\partial f[t,r+W,c+W]}{\partial c}&\frac{\partial f[t,r+W,c+W]}{\partial r}
    \end{array}\right]
  \]
  \[
  \vec{v}=\left[\begin{array}{c}v_c[t,r,c]\\ v_r[t,r,c]\end{array}\right]
  \]
\end{frame}

\begin{frame}
  \frametitle{Pseudo-Inverse}

  The Lucas-Kanade equation cannot be solved exactly, because it is
  $(2W+1)$ equations in only two unknowns.  But we can find the
  minimum-squared error solution, which is

  \[
  v^*[t,r,c] = A^\dag\vec{b} = (A^TA)^{-1}A^T\vec{b}
  \]
\end{frame}


\end{document}

