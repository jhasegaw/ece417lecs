\documentclass{beamer}
\usepackage{tikz,amsmath,amssymb,hyperref,graphicx,stackrel,setspace,animate,listings}
\usetikzlibrary{positioning,shadows,arrows,shapes,calc}
\newcommand{\argmax}{\operatornamewithlimits{argmax}}
\newcommand{\argmin}{\operatornamewithlimits{argmin}}
\mode<presentation>{\usetheme{Frankfurt}}
\DeclareMathOperator*{\softmax}{softmax}
\AtBeginSection[]
{
  \begin{frame}<beamer>
    \frametitle{Outline}
    \tableofcontents[currentsection,currentsubsection]
  \end{frame}
}
\title{Lecture 10: Exam 1 Review}
\author{Mark Hasegawa-Johnson\\These slides are in the public domain}
\date{ECE 417: Multimedia Signal Processing, Fall 2023}  
\begin{document}

% Title
\begin{frame}
  \maketitle
\end{frame}

% Title
\begin{frame}
  \tableofcontents
\end{frame}


%%%%%%%%%%%%%%%%%%%%%%%%%%%%%%%%%%%%%%%%%%%%
\section[Admin]{Test Administration}
\setcounter{subsection}{1}

\begin{frame}
  \frametitle{Test Administration}

  \begin{itemize}
  \item Exam 1 will be in class, Tuesday, 9/26
  \item If you need an online exam or a conflict exam, please contact
    Prof. Hasegawa-Johnson by the end of Friday, 9/22
  \end{itemize}
\end{frame}
  
\begin{frame}
  \frametitle{What to Bring}

  \begin{itemize}
  \item Pencils and erasers
  \item One page of notes, front and back, handwritten or 12pt font
  \item No calculators, computers, cell phones, smart speakers,
    earpods, or other devices capable of communicating with your AI
    or human tutor
  \item Exam will provide a formula sheet---the same one that's on the
    practice exam online
  \end{itemize}
\end{frame}
  
%%%%%%%%%%%%%%%%%%%%%%%%%%%%%%%%%%%%%%%%%%%%
\section[Content]{Test Content}
\setcounter{subsection}{1}

\begin{frame}
  \frametitle{What's on the Exam}

  The exam has four questions:
  \begin{itemize}
  \item One question based on HW1 (linear algebra \& pseudo-inverse)
  \item One question based on MP1 (barycentric coordinates \& bilinear interpolation)
  \item One question based on HW2 (signal processing \& STFT)
  \item One question based on MP2 (Griffin-Lim)
  \end{itemize}
\end{frame}

\begin{frame}
  \frametitle{What's on the Exam}

  \begin{itemize}
  \item Content is from lectures 1-6, including only the material that
    actually made it onto HW1-2 and MP1-2
  \item Lecture 7 is sort of on the exam, and sort of not.  The exam
    will include MMSE affine transforms.  I think the easiest way to
    find the MMSE affine transform is using the derivations in lecture
    7.  So for that reason, I recommend that you also learn lecture
    7's material.
  \end{itemize}
\end{frame}

\begin{frame}
  \frametitle{One question based on HW1 (linear algebra \& pseudo-inverse)}

  \begin{itemize}
  \item Eigenvalues, eigenvectors, singular values \& singular vectors are \textbf{NOT} on the exam,
    because they weren't on the HW
  \item Other things from lecture 1 are covered, e.g., determinant
  \item Algebraic forms of pseudo-inverse will be covered, including:
    \begin{itemize}
    \item MMSE approximation using the pseudo-inverse of a tall thin matrix
    \item Orthogonal projection using the pseudo-inverse of a short fat matrix
    \end{itemize}
  \end{itemize}
\end{frame}

\begin{frame}
  \frametitle{One question based on MP1 (barycentric coordinates \& bilinear interpolation)}

  \begin{itemize}
  \item You should know how to convert cartesian coordinates to/from barycentric coordinates,
    and how to tell which triangle contains a point
  \item You should know how to use bilinear interpolation to find the
    color of a point between the input points
  \end{itemize}
\end{frame}

\begin{frame}
  \frametitle{One question based on HW2 (signal processing \& STFT)}

  \begin{itemize}
  \item Understand DTFT and DFT, impulses in both time and frequency,
    rectangles in both time and frequency
  \item Know that Hamming, Hann \& Bartlett windows have a main lobe
    twice as wide as a rectangular window, with much lower sidelobes
  \item Understand forward STFT, and OLA method of inverse STFT,
    including reasons why $\sum_m w[n-m]$ is usually a constant (and
    cases in which the constant might not be 1)
  \end{itemize}
\end{frame}

\begin{frame}
  \frametitle{One question based on MP2 (Griffin-Lim)}

  \begin{itemize}
  \item Conjugate-symmetry constraint, overlapped-samples constraint
  \item Zero-phase, random phase, correct phase
  \item Understand the interpretation of Griffin-Lim as a sequence of
    orthogonal projections
  \item Understand Griffin-Lim as a sequence of angle(STFT(ISTFT))
    operations
  \end{itemize}
\end{frame}

%%%%%%%%%%%%%%%%%%%%%%%%%%%%%%%%%%%%%%%%%%%%
\section[Sample Problems]{Sample Problems}
\setcounter{subsection}{1}

\begin{frame}
  \frametitle{Sample Problems}

  \begin{itemize}
  \item You are welcome to do any sample problems you find online,
    including any previous exams from this course.
  \item I've collected the problems I consider most relevant into the
    online practice exam.
  \item It's much longer than the real exam will be.
  \end{itemize}
\end{frame}

\end{document}

