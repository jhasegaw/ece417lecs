


\begin{frame}
  \frametitle{DTFT of a Windowed Sinusoid}

  Suppose we just chop  out $N$ samples of that sinusoid:
  \[
  x[n]=\begin{cases}
  A\cos\left(\omega_0 n+\theta\right) & 0\le n\le N-1\\
  0 & \mbox{otherwise}
  \end{cases}
  \]
  Then its DTFT is:
  \[
  X(\omega) = \frac{A}{2}e^{j\theta}W_R(\omega-\omega_0)+\frac{A}{2}e^{-j\theta}W_R(\omega+\omega_0)
  \]
  where $W_R(\omega)$ is the DTFT of the rectangular window,
  \[
  W_R(\omega) = e^{-j\omega\left(\frac{N-1}{2}\right)}\frac{\sin(\omega N/2)}{\sin(\omega/2)}
  \]
\end{frame}

\begin{frame}
  \frametitle{DFT of a Windowed Sinusoid}

  Remember that the DFT is defined to be
  \[
  X[k]=\sum_{n=0}^{N-1}x[n]e^{-j\left(\frac{2\pi kn}{N}\right)},~~~~
  x[n]=\frac{1}{N}\sum_{k=0}^{N-1}X[k]e^{j\left(\frac{2\pi kn}{N}\right)}
  \]
  So the DFT of a windowed cosine is:
  \[
  X[k] = \frac{A}{2}e^{j\theta}W_R\left(\frac{2\pi k}{N}-\omega_0\right)+
  \frac{A}{2}e^{-j\theta}W_R\left(\frac{2\pi k}{N}+\omega_0\right)
  \]
\end{frame}

\begin{frame}
  \frametitle{DFT of a Windowed Sinusoid}
  \centerline{\includegraphics[height=1.5in]{exp/tone_white_waveform.png}}
  \centerline{\includegraphics[height=1.5in]{exp/tone_white_powerspectrum.png}}
\end{frame}

\begin{frame}
  \frametitle{DFT of a Windowed Sinusoid}
  \[
  X[k] = \frac{A}{2}e^{j\theta}W_R\left(\frac{2\pi k}{N}-\omega_0\right)+
  \frac{A}{2}e^{-j\theta}W_R\left(\frac{2\pi k}{N}+\omega_0\right)
  \]
  \begin{itemize}
  \item If this formula is so complicated, how come the results on the
    previous slide are so simple?
  \item Answer: the window (20ms=160 samples) is an integer multiple of
    the period (1ms=8 samples), thus $N=k_0 N_0$, and $\omega_0=\frac{2\pi k_0}{N}$.
  \item When $\omega_0=\frac{2\pi k_0}{N}$, then
    \[
    W_R\left(\frac{2\pi k}{N}-\omega_0\right) = W_R\left(\frac{2\pi (k-k_0)}{N}\right)
    =
    \frac{\sin\left(\frac{2\pi (k-k_0)}{N}\frac{N}{2}\right)}{\sin\left(\frac{2\pi (k-k_0)}{N}\frac{1}{2}\right)}
    =\begin{cases}
    N & k=k_0\\
    0 & \mbox{otherwise}
    \end{cases}
    \]
  \end{itemize}
\end{frame}

\begin{frame}
  \frametitle{DFT of a Sinusoid (Easy Case)}
  So if $\omega_0=\frac{2\pi k_0}{N}$, and 
  \[
  x[n] = A\cos(\omega_0 n+\theta)
  \]
  then
  \[
  X[k] = \frac{ANe^{j\theta}}{2}\delta(k-k_0)+\frac{ANe^{-j\theta}}{2}\delta(k+k_0)
  \]
  and
  \[
  |X[k]|^2 = \frac{(AN)^2}{4}\delta(k-k_0)+\frac{(AN)^2}{4}\delta(k+k_0)
  \]
\end{frame}

\begin{frame}
  \frametitle{DFT Power Spectrum of a Pure Tone}
  \[
  |X[k]|^2 = \frac{(AN)^2}{4}\delta(k-k_0)+\frac{(AN)^2}{4}\delta(k+k_0)
  \]
  \centerline{\includegraphics[height=1.5in]{exp/tone_white_waveform.png}}
  \centerline{\includegraphics[height=1.5in]{exp/tone_white_powerspectrum.png}}
\end{frame}

\begin{frame}
  \frametitle{Parseval's Theorem for a Pure Tone (Easy Case)}

  The energy in the time domain is $N$ times the power.  The power of
  a sinusoid is $A^2/2$.  So the energy is:
  \[
  \sum_{n=0}^{N-1} A^2\cos^2\left(\omega_0 N+\theta\right) = \frac{NA^2}{2}
  \]
  The energy in the frequency domain is:
  \[
  \frac{1}{N}\sum_{k=0}^{N-1} |X[k]|^2 =
  \frac{1}{N}\left(\frac{(AN)^2}{4}+\frac{(AN)^2}{4}\right) =\frac{NA^2}{2}
  \]
\end{frame}

\begin{frame}
  \frametitle{Parseval's Theorem for a Pure Tone (Hard Case)}

  The energy in the time domain is:
  \[
  \sum_{n=0}^{N-1} A^2\cos^2\left(\omega_0 N+\theta\right) = \frac{NA^2}{2}
  \]
  The energy in the frequency domain is:
  \[
  \frac{1}{N}\sum_{k=0}^{N-1} |X[k]|^2 =
  \frac{A^2}{4N}\sum_{k=0}^{N-1}\left|
  e^{j\theta}W_R\left(\frac{2\pi k}{N}-\omega_0\right)+
  e^{-j\theta}W_R\left(\frac{2\pi k}{N}+\omega_0\right)\right|^2
  \]
  By setting the bottom equation equal to the top equation, we discover that
  the two rectangular windows have a total energy of $2N^2$, totally regardless of the values
  of $\omega_0$ and $\theta$.  So for all practical purposes, the ``Hard Case''
  is just as easy as the ``Easy Case,'' and we can ignore the hard part.
\end{frame}
