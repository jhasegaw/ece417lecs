\documentclass{beamer}
\usepackage{listings}
\usepackage{hyperref}
\usepackage{tikz}
\usetikzlibrary{positioning,shadows,arrows,shapes,calc}
\def\labelenumi\theenumi
\usepackage{graphicx}
\usepackage{amsmath}
\mode<presentation>{\usetheme{Frankfurt}}
\AtBeginSection
{
  \begin{frame}<beamer>
    \frametitle{Outline}
    \tableofcontents[currentsection,currentsubsection]
  \end{frame}
}
\title{Lecture 20: Rotating, Scaling, Shifting and Shearing an Image}
\author{Mark Hasegawa-Johnson\\All content~\href{https://creativecommons.org/licenses/by-sa/4.0/}{CC-SA 4.0} unless otherwise specified.}
\date{ECE 417: Multimedia Signal Processing, Fall 2020}  
\institute{University of Illinois}
\titlegraphic{\includegraphics{../../../17fall/lectures/imark_1867_bold.png}}
\begin{document}

% Title
\begin{frame}
  \maketitle
\end{frame}

% Title
\begin{frame}
  \tableofcontents
\end{frame}

%%%%%%%%%%%%%%%%%%%%%%%%%%%%%%%%%%%%%%%%%%%%%%%%%%%%%%%%%
\section{Modifying an Image by Moving Its Points}
\setcounter{subsection}{1}

\begin{frame}
  \frametitle{Moving Points Around}
  First, let's suppose that somebody has given you a bunch of points:
  \centerline{\includegraphics[width=4in]{../../../17fall/lectures/mp7_image006.jpg}}
\end{frame}

\begin{frame}
  \begin{columns}[t]
    \column{1in}
    \begin{block}{}
      \ldots and let's suppose you want to move them around, to create
      new images\ldots
    \end{block}
    \column{4in}
    \begin{block}{}
      \centerline{\includegraphics[width=4in]{../../../17fall/lectures/huang_face_tracking.png}}
    \end{block}
  \end{columns}
\end{frame}

\begin{frame}
  \frametitle{Moving One Point}
  \begin{itemize}
  \item Your goal is to synthesize an output image, $J[y,x]$, where
    $J[y,x]$ might be intensity, or RGB vector, or whatever, $y$ is
    {\bf row} number (measured from top to bottom), $x$ is {\bf
      column} number (measured from left to right).
  \item What you have available is:
    \begin{itemize}
    \item An input image, $I[n,m]$, sampled at integer values of $m$ and $n$.
    \item Knowledge that the input point at $I(v,u)$ has been {\bf
      moved} to the output point at $J[y,x]$, where $x$ and $y$ are
      integers, but $u$ and $v$ might not be integers.
    \end{itemize}
  \end{itemize}
  \[
  J[y,x] = I(v,u)
  \]
\end{frame}

%%%%%%%%%%%%%%%%%%%%%%%%%%%%%%%%%%%%%%%%%%%%%%%%%%%%%%%%%
\section{Affine Transformations}
\setcounter{subsection}{1}

\begin{frame}
  \frametitle{How do we find $(u,v)$?}
  Now the question: how do we find $(u,v)$?

  For today, let's assume that this is a piece-wise affine transformation.
  \[
  \left[\begin{array}{c} u\\v\end{array}\right]=
  \left[\begin{array}{cc}a&b\\d&e\end{array}\right]
  \left[\begin{array}{c}x\\y\end{array}\right] +
  \left[\begin{array}{c}c\\f\end{array}\right]
  \]
\end{frame}
  
\begin{frame}
  \frametitle{How do we find $(u,v)$?}
  An affine transformation is defined by:
  \[
  \left[\begin{array}{c} u\\v\end{array}\right]=
  \left[\begin{array}{cc}a&b\\d&e\end{array}\right]
  \left[\begin{array}{c}x\\y\end{array}\right] +
  \left[\begin{array}{c}c\\f\end{array}\right]
  \]
  A much easier to write this is by using extended-vector notation:
  \[
  \left[\begin{array}{c} u\\v\\1\end{array}\right]=
  \left[\begin{array}{ccc}a&b&c\\d&e&f\\0&0&1\end{array}\right]
  \left[\begin{array}{c}x\\y\\1\end{array}\right]
  \]
  It's convenient to define $\vec{u}=[u,v,1]^T$, and
  $\vec{x}=[x,y,1]^T$, so that for any $\vec{x}$ in the
  output image,
  \[
  \vec{u}=A \vec{x}
  \]
\end{frame}
  
\begin{frame}
  \frametitle{Affine Transforms}

  Notice that the affine transformation has 6 degrees of freedom:
  $(a,b,c,d,e,f)$.  Therefore, you can accmplish 6 different types of
  transformation:
  \begin{itemize}
  \item Shift the image left$\leftrightarrow$right (using $c$)
  \item Shift the image up$\leftrightarrow$down (using $f$)
  \item Scale the image horizontally (using $a$)
  \item Scale the image vertically (using $e$)
  \item Rotate the image (using $a,b,d,e$)
  \item Shear the image horizontally (using $b$)
  \end{itemize}
  Vertical shear (using $d$) is a combination of horizontal shear + rotation.
\end{frame}

\begin{frame}
  \frametitle{Example: Reflection}
  \centerline{\includegraphics[height=2in]{../../../17fall/lectures/Affine_Transformation_Original_Checkerboard.jpg}\includegraphics[height=2in]{../../../17fall/lectures/Affine_Transformation_Reflected_Checkerboard.jpg}}
  \[
  \left[\begin{array}{c} u\\v\\1\end{array}\right]=
  \left[\begin{array}{ccc}-1&0&0\\0&1&0\\0&0&1\end{array}\right]
  \left[\begin{array}{c}x\\y\\1\end{array}\right]
  \]
\end{frame}

\begin{frame}
  \frametitle{Example: Scale}
  \centerline{\includegraphics[height=2in]{../../../17fall/lectures/Affine_Transformation_Original_Checkerboard.jpg}\includegraphics[height=2in]{../../../17fall/lectures/Affine_Transformation_Scale_Checkerboard.jpg}}
  \[
  \left[\begin{array}{c} u\\v\\1\end{array}\right]=
  \left[\begin{array}{ccc}2&0&0\\0&1&0\\0&0&1\end{array}\right]
  \left[\begin{array}{c}x\\y\\1\end{array}\right]
  \]
\end{frame}

\begin{frame}
  \frametitle{Example: Rotation}
  \centerline{\includegraphics[height=2in]{../../../17fall/lectures/Affine_Transformation_Original_Checkerboard.jpg}\includegraphics[height=2in]{../../../17fall/lectures/Affine_Transformation_Rotated_Checkerboard.jpg}}
  \[
  \left[\begin{array}{c} u\\v\\1\end{array}\right]=
  \left[\begin{array}{ccc}\cos\theta&-\sin\theta&0\\\sin\theta&\cos\theta&0\\0&0&1\end{array}\right]
  \left[\begin{array}{c}x\\y\\1\end{array}\right]
  \]
\end{frame}

\begin{frame}
  \frametitle{Example: Shear}
  \centerline{\includegraphics[height=2in]{../../../17fall/lectures/Affine_Transformation_Original_Checkerboard.jpg}\includegraphics[height=2in]{../../../17fall/lectures/Affine_Transformation_Shear_Checkerboard.jpg}}
  \[
  \left[\begin{array}{c} u\\v\\1\end{array}\right]=
  \left[\begin{array}{ccc}1&0.5&0\\0&1&0\\0&0&1\end{array}\right]
  \left[\begin{array}{c}x\\y\\1\end{array}\right]
  \]
\end{frame}

\begin{frame}
  \centerline{\href{https://www.youtube.com/watch?v=il6Z5LCykZk}{\includegraphics[width=4.5in]{../../../17fall/lectures/youtube_affine.png}}}
\end{frame}

%%%%%%%%%%%%%%%%%%%%%%%%%%%%%%%%%%%%%%%%%%%%%%%%%%%%%%%%%
\section{Image Interpolation}
\setcounter{subsection}{1}

\begin{frame}
  \begin{block}{Integer Output Points}
    Now let's suppose that you've figured out the coordinate transform:
    for any given $J[y,x]$, you've figured out which pixel should be used to create it 
    ($J[y,x]=I(v,u)$).
    \lstinputlisting[language=Python]{interpolation_example.py}
  \end{block}
  \begin{block}{The Problem: Non-Integer Input Points}
    If $[x,y]$ are integers, then usually, $(u,v)$ are not integers.
  \end{block}
\end{frame}

\begin{frame}
  \frametitle{Image Interpolation}
  The function compute\_pixel performs image interpolation.
  Given the pixels of $I[n,m]$ at integer values of $m$ and $n$, it computes
  the pixel at a non-integer position $I(v,u)$ as:
  \[
  I(v,u) = \sum_m\sum_n I[n,m] h(v-n,u-m)
  \]
\end{frame}

\begin{frame}
  \frametitle{Piece-Wise Constant Interpolation}
  \begin{equation}
  I(v,u) = \sum_m\sum_n I[n,m] h(v-n,u-m)
  \label{eq:interpolation1}
  \end{equation}
  For example, suppose
  \[
  h(v,u) = \left\{\begin{array}{ll}
  1 & 0\le u<1,~~0\le v<1\\
  0 & \mbox{otherwise}
  \end{array}\right.
  \]
  Then Eq. (\ref{eq:interpolation1}) is the same as just truncating $u$
  and $v$ to the next-lower integer, and outputting that number:
  \[
  I(v,u) = I\left[\lfloor v\rfloor,\lfloor u\rfloor\right]
  \]
  where $\lfloor u\rfloor$ means ``the largest integer smaller than $u$''.
\end{frame}

\begin{frame}
  \frametitle{Example: Original Image}
  For example, let's downsample this image, and then try to recover it by image interpolation:
  \centerline{\includegraphics[width=4.5in]{exp/original.png}}
\end{frame}

\begin{frame}
  \frametitle{Example: Downsampled Image}
  Here's the downsampled image:
  \centerline{\includegraphics[width=4.5in]{exp/downsampled.png}}
\end{frame}

\begin{frame}
  \frametitle{Example: Upsampled Image} Here it is after we upsample
  it back to the original resolution (insert 3 zeros between every
  pair of nonzero columns):
  \centerline{\includegraphics[width=4.5in]{exp/upsample.png}}
\end{frame}

\begin{frame}
  \frametitle{Example: PWC Interpolation}

  Here is the piece-wise constant interpolated image:
  \centerline{\includegraphics[width=4.5in]{exp/pwc.png}}
\end{frame}

\begin{frame}
  \frametitle{Bi-Linear Interpolation}
  \[
  I(v,u) = \sum_m\sum_n I[n,m] h(v-n,u-m)
  \]
  For example, suppose
  \[
  h(v,u) = \max\left(0,(1-|u|)(1-|v|)\right)
  \]
  Then Eq. (\ref{eq:interpolation1}) is the same as piece-wise linear
  interpolation among the four nearest pixels.  This is called {\bf
    bilinear interpolation} because it's linear in two directions.
  \begin{align*}
    m &= \lfloor u\rfloor,~~~e=u-m\\
    n &= \lfloor v\rfloor,~~~f=v-m\\
    I(v,u) &= (1-e)(1-f)I[n,m]+(1-e)fI[n,m+1]\\
    &+e(1-f)I[n+1,m]+efI[n+1,m+1]
  \end{align*}
\end{frame}

\begin{frame}
  \frametitle{Example: Upsampled Image}

  Here's the upsampled image again:
  \centerline{\includegraphics[width=4.5in]{exp/upsample.png}}
\end{frame}

\begin{frame}
  \frametitle{Example: Bilinear Interpolation}

  Here it is after bilinear interpolation:
  \centerline{\includegraphics[width=4.5in]{exp/pwl.png}}
\end{frame}


\begin{frame}
  \frametitle{PWC and PWL Interpolator Kernels}

  Bilinear interpolation uses a PWL interpolation kernel, which does
  not have the abrupt discontiuity of the PWC interpolator kernel.
  \centerline{\includegraphics[width=4.5in]{exp/interpolators.png}}
\end{frame}


\begin{frame}
  \frametitle{Sinc Interpolation}
  \[
  I(v,u) = \sum_m\sum_n I[n,m] h(v-n,u-m)
  \]
  For example, suppose
  \[
  h(v,u) = \mbox{sinc}(\pi u)\mbox{sinc}(\pi v)
  \]
  Then Eq. (\ref{eq:interpolation1}) is an ideal band-limited sinc interpolation.
  It guarantees that the continuous-space image, $I(v,u)$, is exactly a band-limited
  D/A reconstruction of the digital image $I[n,m]$.
\end{frame}

\begin{frame}
  \frametitle{Sinc Interpolation}

  Here is the cat after sinc interpolation:
  \centerline{\includegraphics[width=4.5in]{exp/sincinterp.png}}
\end{frame}

\begin{frame}
  \frametitle{Original, Upsampled, and Sinc-Interpolated Spectra}

  Here are the magnitude Fourier transforms of the original,
  upsampled, and sinc-interpolated cat.
  
  \centerline{\includegraphics[width=4.5in]{exp/downsampled_spectra.png}}
\end{frame}

\begin{frame}
  \frametitle{Original, Upsampled, and Sinc-Interpolated Spectra}

  Here are the magnitude Fourier transforms of the original,
  upsampled, and sinc-interpolated cat.

  \centerline{
    \begin{tikzpicture}
      \node[inner sep=0pt](spectra) at (0,1.5) {
        \includegraphics[height=5cm]{exp/downsampled_spectra.png}};
      \node[inner sep=0pt](orig) at (4.5,2.8) {
        \includegraphics[height=1.5cm]{exp/original.png}};
      \node[inner sep=0pt](up) at (4.5,1.5) {
        \includegraphics[height=1.5cm]{exp/upsample.png}};
      \node[inner sep=0pt](sinc)  at (4.5,0.2) {
        \includegraphics[height=1.5cm]{exp/sincinterp.png}};
  \end{tikzpicture}}
  \vspace*{-0.5cm}
  \begin{itemize}
  \item The zeros in the upsampled cat correspond to aliasing in its spectrum.
  \item The ringing in the sinc-interpolated cat corresponds to the
    sharp cutoff, at pi/4, of its spectrum.
  \end{itemize}
\end{frame}


%%%%%%%%%%%%%%%%%%%%%%%%%%%%%%%%%%%%%%%%%%%%%%%%%%%%%%%%%
\section{Conclusions}
\setcounter{subsection}{1}
\begin{frame}
  \frametitle{Conclusions}
  \begin{itemize}
  \item You can generate an output image $J[y,x]$ by warping an input image $I(v,u)$.
  \item If $(v,u)$ are not integers, you can compute the value of $I(v,u)$ by interpolating
    among $I[n,m]$, where $[n,m]$ are integers.
    \[
    I(v,u) = \sum_m\sum_n I[n,m] h(v-n,u-m)
    \]
  \item Shift, scale, rotation and shear are affine transformations, given by
  \[
  \left[\begin{array}{c} u\\v\\1\end{array}\right]=
  \left[\begin{array}{ccc}a&b&c\\d&e&f\\0&0&1\end{array}\right]
  \left[\begin{array}{c}x\\y\\1\end{array}\right]
  \]
  \end{itemize}
\end{frame}
\end{document}

