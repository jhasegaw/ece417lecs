  \begin{center}
    \tikzstyle{post}=[->,shorten <=1pt,>=stealth',semithick,draw=blue]
    \tikzstyle{state}=[circle,thin,draw=blue,text width=0.25cm,fill=white]
    \tikzstyle{initial}=[circle,thick,draw=blue,text width=0.25cm,fill=white]
    \tikzstyle{final}=[circle,double,draw=blue,text width=0.25cm,fill=white]
    \begin{tikzpicture}[->,>=stealth',shorten >=1pt,auto,node distance=3cm,thick]
      \node[initial](q1)at(0,0){1};
      \node[state](q2)at(1.5,0){2};
      \node[state](q3)at(3,0){3};
      \node[state](q4)at(4.5,0){4};
      \node[state](q5)at(6,0){5};
      \node[final](q6)at(8,0.5){6};
      \node[state](q7)at(8,-0.5){7};
      \node[final](q8)at(10,0){8};
      \node[final](q9)at(10,-1){9};
      \path[every node/.style={font=\sffamily\small,fill=white,inner sep=1pt}]
      (q1)edge[out=45,in=135] node {\ipa{I}}(q2)
      edge[out=-45,in=-135] node {\ipa{i}}(q2)
      (q2)edge node {\ipa{l}}(q3)
      (q3)edge[in=135,out=45] node {\ipa{I}} (q4)
      edge[in=-135,out=-45] node {\ipa{i}} (q4)
      (q4) edge node {\ipa{n}} (q5)
      (q5) edge[out=45,in=180] node {\ipa{OI}} (q6)
      edge[out=-45,in=180] node {\ipa{w}} (q7)
      (q7) edge[out=45,in=180] node {\ipa{O}} (q8)
      edge[out=-45,in=180] node {\ipa{e}} (q9);
    \end{tikzpicture}
  \end{center}
    

  \begin{frame}
  \frametitle{The Log Semiring}

  \begin{itemize}
  \item If $p(A,B)=p(A|B)p(B)$, then $c(A,B)=c(A|B)+c(B)$.  So we can
    say that in the log semiring, the $\otimes$ operator is $\otimes=+$.
  \item If $p\cdot 1=p$, then $c+ (-\ln(1)) = c$.  So in this
    semiring, $\bar{1}=0$.
  \item We have the start of a semiring:
    $\left\{\mathbb{P},?,+,?,0\right\}$.  The question mark denotes
    the thing we don't know yet: what is the $\oplus$ operation in the
    log semiring?
  \end{itemize}
\end{frame}

\begin{frame}
  \frametitle{Oplus in the Log Semiring: the Logsumexp function}

  The oplus operation in the log semiring is the {\bf neglogsumexp}
  function:
  \begin{displaymath}
    a\oplus b=\neglogsumexp(a,b) = -\ln\left(e^{-a}+e^{-b}\right)
  \end{displaymath}
  The equation above is the shortest way to write this function, but
  not the most numerically stable way.  If you're worried about
  floating point underflow, it is possible to write this function in
  a way that guarantees that it never underflows:
  \begin{align*}
    m &= \min(a,b)\\
    a\oplus b &= m -\ln\left(e^{m-a}+e^{m-b}\right)
  \end{align*}
\end{frame}

\begin{frame}
  \frametitle{The Log Semiring}

  The log semiring is $\left\{\mathbb{P},\neglogsumexp,+,\infty,0\right\}$.
  \begin{itemize}
  \item The set of all negative log probabilities and densities is
    $\mathbb{P}=\mathbb{R}\cup\infty$.
  \item $\otimes=+$: $p_1=p_2\cdot p_3\Rightarrow c_1=c_2\otimes c_3$.
  \item $\bar{1}=0$: $p_1=p_2\cdot 1\Rightarrow c_1=c_2\otimes\bar{1}$
  \item The $\oplus$ operator is $\oplus=\neglogsumexp$:
    $p_1+p_2=p_3 \Rightarrow c_1\oplus c_2 =c_3$.
  \item The identity element for the $\oplus$ operator is $\infty$:
    \begin{align*}
      m &= \min(a,\infty)  = a\\
      a\oplus \infty &= a -\ln\left(e^{a-a}+e^{a-\infty}\right) = a
    \end{align*}
  \end{itemize}
\end{frame}

\end{document}

%%%%%%%%%%%%%%%%%%%%%%%%%%%%%%%%%%%%%%%%%%%%
\section[Semirings]{Semirings}
\setcounter{subsection}{1}

\begin{frame}
  \frametitle{Semirings}

  A {\bf semiring},
  $K=\left\{\mathbb{K},\oplus,\otimes,\bar{0},\bar{1}\right\}$, is a
  set $\mathbb{K}$ that is closed under two associative operators
  $\oplus$ and $\otimes$. Each of the operators has its own identity
  element, $\bar{0}$ and $\bar{1}$ ($a\oplus\bar{0}=a$,
  $a\otimes\bar{1}=a$), and:
  \begin{itemize}
  \item $\otimes$ distributes over $\oplus$,
  \item $a\otimes\bar{0}=\bar{0}$.
  \end{itemize}
\end{frame}

\begin{frame}
  \frametitle{The Natural Numbers Semirings}

  For example, the natural numbers
  $\mathbb{N}=\left\{0,1,2,\ldots\right\}$ are closed under regular
  addition and multiplication:
  \begin{itemize}
  \item If $a$ and $b$ are natural numbers, then so is $a+b$,
  \item If $a$ and $b$ are natural numbers, then so is $a\cdot b$,
  \item If $a$ is a natural number, then $a+0=a$
  \item If $a$ is a natural number, then $a\cdot 1=a$
  \item If $a$ is a natural number, then $a\cdot 0=0$
  \item If $a$, $b$, and $c$ are natural numbers, then $a\cdot (b+c) = ab+ac$
  \end{itemize}
  so $\left\{\mathbb{N},+,\cdot,0,1\right\}$ is a semiring.
\end{frame}

\begin{frame}
  \frametitle{The Probability Semiring}

  Probability densities are numbers in the set $\mathbb{R}_+$ (the
  non-negative real numbers).  This set is closed over multiplication
  and addition, so
  \begin{itemize}
  \item If $a$ and $b$ are probability densities, then so is $a+b$,
  \item If $a$ and $b$ are probability densities, then so is $a\cdot b$,
  \item If $a$ is a probability density, then $a+0=a$
  \item If $a$ is a probability density, then $a\cdot 1=a$
  \item If $a$ is a probability density, then $a\cdot 0=0$
  \item If $a$, $b$, and $c$ are probability densities, then $a\cdot (b+c) = ab+ac$
  \end{itemize}
  so $\left\{\mathbb{R}_+,+,\cdot,0,1\right\}$ is a semiring.
\end{frame}

\begin{frame}
  \frametitle{The Log Semiring}

  Surprisals are numbers in the set $\bar\mathbb{R}$ (the so-called
  ``extended reals,''
  $\bar\mathbb{R}=\mathbb{R}\cup\left\{-\infty,+\infty\right\}$).
  This set is closed over addition and negsumexp, so
  \begin{itemize}
  \item If $a$ and $b$ are surprisals, then so is $a\otimes b$,
  \item If $a$ and $b$ are surprisals, then so is $a\oplus b$,
  \item If $a$ is a surprisal, then $a\oplus \infty=a$
  \item If $a$ is a surprisal, then $a+0=a$
  \item If $a$ is a probability density, then $a\cdot 0=0$
  \item If $a$, $b$, and $c$ are probability densities, then $a\cdot (b+c) = ab+ac$
  \end{itemize}
  so $\left\{\mathbb{R}_+,+,\cdot,0,1\right\}$ is a semiring.
\end{frame}

\begin{frame}
  \frametitle{The Probability Semiring}

  Probabilities and probability densities are numbers in the set
  $\mathbb{R}_+$ (the non-negative real numbers).  This set is closed
  over multiplication and addition, so
  $\left\{\mathbb{R}_+,+,\cdot,0,1\right\}$ is a semiring called the
  ``probability semiring.''
  \begin{itemize}
  \item Marginalization: $p(B=b)= \sum_a p(B=b,A=a)$
    \begin{itemize}
    \item Identity element: $p+0=p$.
    \end{itemize}
  \item Multiplication of conditional probabilities: $p(A,B)=p(A)p(B|A)$
    \begin{itemize}
    \item Identity element: $p\cdot 1=p$
    \item Multiplication by zero: $p\cdot 0=0$
    \end{itemize}
  \end{itemize}
\end{frame}

\begin{frame}
  \frametitle{The Log Semiring}

  \begin{itemize}
  \item In order to avoid floating-point underflow, it's often useful
    to work with negative log-probabilities, usually called
    ``surprisals'' or ``costs:''
    \[
    c(A|B) = -\ln p(A|B)
    \]
  \item If $p(A|B)\in\mathbb{R}_+$, then
    $c(A|B)\in\mathbb{R}\cup\left\{\infty\right\}$.
  \item The set $\mathbb{R}\cup\left\{-\infty,\infty\right\}$ is
    called the ``extended real'' set, and is usually denoted
    $\bar\mathbb{R}$.  We don't really need $-\infty$, but we really,
    really need $+\infty$ (remember, $-\ln(0)=+\infty$), so we can say
    that surprisals come from the set $\bar\mathbb{R}$.
  \end{itemize}
\end{frame}


    \end{align*}
    \end{align*}
