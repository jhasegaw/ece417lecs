\documentclass{beamer}
\usepackage{tikz,amsmath,hyperref,graphicx,stackrel,animate,tipa,tcolorbox}
\usetikzlibrary{positioning,shadows,arrows,shapes,calc}
\newcommand{\ipa}[1]{\fontfamily{cmr}\selectfont\textipa{#1}}
\newcommand{\best}{\operatornamewithlimits{best}}
\newcommand{\argbest}{\operatornamewithlimits{argbest}}
\newcommand{\argmax}{\operatornamewithlimits{argmax}}
\newcommand{\argmin}{\operatornamewithlimits{argmin}}
\newcommand{\logsumexp}{\operatornamewithlimits{logsumexp}}
\mode<presentation>{\usetheme{Frankfurt}}
\DeclareMathOperator*{\softmax}{softmax}
\AtBeginSection[]
{
  \begin{frame}<beamer>
    \frametitle{Outline}
    \tableofcontents[currentsection,currentsubsection]
  \end{frame}
}
\title{Lecture 16: Weighted Finite State Transducers (WFST)}
\author{Mark Hasegawa-Johnson\\All content~\href{https://creativecommons.org/licenses/by-sa/4.0/}{CC-SA 4.0} unless otherwise specified.}
\date{ECE 417: Multimedia Signal Processing, Fall 2020}  
\begin{document}

% Title
\begin{frame}
  \maketitle
\end{frame}

% Title
\begin{frame}
  \tableofcontents
\end{frame}

%%%%%%%%%%%%%%%%%%%%%%%%%%%%%%%%%%%%%%%%%%%%
\section[Review]{Review: WFSA}
\setcounter{subsection}{1}

\begin{frame}
  \frametitle{Weighted Finite State Acceptors}
  \begin{center}
    \tikzstyle{state}=[circle,thin,draw=blue,text width=0.25cm,fill=white]
    \tikzstyle{initial}=[circle,thick,draw=blue,text width=0.25cm,fill=white]
    \tikzstyle{final}=[circle,double,draw=blue,text width=0.25cm,fill=white]    
    \begin{tikzpicture}[->,>=stealth',shorten >=1pt,auto,node distance=3cm,thick]
      \node[initial] (q0) at (0,0) {0};
      \node[state] (q1) at (2,1) {1};
      \node[state] (q2) at (2,-1) {2};
      \node[state] (q3) at (4,0) {3};
      \node[state] (q4) at (6,0) {4};
      \node[final] (q5) at (8,0) {5};
      \node[final] (q6) at (8,-1) {6};
      \path[every node/.style={font=\sffamily\small,
  	  fill=white,inner sep=1pt}]
      (q0) edge [out=60,in=135] node {The/$0.3$} (q1)
      edge [out=30,in=-135] node {A/$0.2$} (q1)
      edge [out=-45,in=180] node {A/$0.3$} (q2)
      edge [out=-135,in=-135] node {This/$0.2$} (q2)
      (q1) edge [out=0,in=135] node {dog/$1$} (q3)
      (q2) edge [out=45,in=180] node {dog/$0.3$} (q3)
      edge [out=-45,in=-90] node {cat/$0.7$} (q3)
      (q3) edge node {is/$1$} (q4)
      (q4) edge [out=120,in=60,looseness=4] node {very/$0.2$} (q4)
      edge [out=0,in=180] node {cute/$0.4$} (q5)
      edge [out=-45,in=180] node {hungry/$0.4$} (q6);
    \end{tikzpicture}
  \end{center}
  
  \begin{itemize}
  \item An {\bf FSA} specifies a set of strings.  A string is in the set if
    it corresponds to a valid path from start to end, and not otherwise.
  \item A {\bf WFSA} also specifies a probability mass function over the set.
  \end{itemize}
\end{frame}

\begin{frame}
  \frametitle{Every Markov Model is a WFSA}
  \begin{center}
    \tikzstyle{state}=[circle,thin,draw=blue,text width=0.25cm,fill=white]
    \tikzstyle{initial}=[circle,thick,draw=blue,text width=0.25cm,fill=white]
    \tikzstyle{final}=[circle,double,draw=blue,text width=0.25cm,fill=white]    
    \begin{tikzpicture}[->,>=stealth',shorten >=1pt,auto,node distance=3cm,thick]
      \node[initial] (q1) at (0,0) {1};
      \node[state] (q2) at (3,0) {2};
      \node[final] (q3) at (6,0) {3};
      \path[every node/.style={font=\sffamily\small,
  	  fill=white,inner sep=1pt}]
      (q1) edge [out=-120,in=120,looseness=4] node {1/$a_{11}$} (q1)
      edge [out=15,in=165] node {1/$a_{12}$} (q2)
      edge [out=45,in=135] node {1/$a_{13}$} (q3)
      (q2) edge [out=120,in=60,looseness=4] node {2/$a_{22}$} (q2)
      edge [out=-165,in=-15] node {2/$a_{21}$} (q1)
      edge [out=15,in=165] node {2/$a_{23}$} (q3)
      (q3) edge [out=60,in=-60,looseness=4] node {3/$a_{33}$} (q3)
      edge [out=-165,in=-15] node {3/$a_{32}$} (q2)
      edge [out=-135,in=-45] node {3/$a_{31}$} (q1);
    \end{tikzpicture}
  \end{center}
  A Markov Model (but not an HMM!) may be interpreted as a WFSA: just
  assign a label to each edge.  The label might just be the state
  number, or it might be something more useful.
\end{frame}
  
\begin{frame}
  \frametitle{Best-Path Algorithm for a WFSA}

  Given:
  \begin{itemize}
  \item Input string, $S=[s_1,\ldots,s_T]$.  For example, the
    string ``A dog is very very hungry'' has $T=5$ words.
  \item Edges, $e$, each have predecessor state $p[e]\in Q$, next state
    $n[e]\in Q$, weight $w[e]\in\overline{\mathbb{R}}$ and label $\ell[e]\in\Sigma$.
  \end{itemize}
  \begin{itemize}
  \item {\bf Initialize:}
    \begin{displaymath}
      \delta_0(i) = \begin{cases}
        \bar{1} & i=\mbox{initial state}\\
        \bar{0} & \mbox{otherwise}
      \end{cases}
    \end{displaymath}
  \item {\bf Iterate:}
    \begin{align*}
      \delta_t(j) &= \best_{e:n[e]=j,\ell[e]=s_t} \delta_{t-1}(p[e]) \otimes w[e]\\
      \psi_t(j) &= \argbest_{e:n[e]=j,\ell[e]=s_t} \delta_{t-1}(p[e]) \otimes w[e]
    \end{align*}
  \item {\bf Backtrace:}
    \begin{displaymath}
      e^*_t = \psi(q^*_{t+1}),~~~~~q^*_t=p[e^*_t]
    \end{displaymath}
  \end{itemize}
\end{frame}

\begin{frame}
  \frametitle{Determinization}

  A WFSA is said to be {\bf deterministic} if, for any given
  (predecessor state $p[e]$, label $\ell[e]$), there is at most 
  one such edge.  For example, this WFSA is not deterministic.

  \begin{center}
    \tikzstyle{state}=[circle,thin,draw=blue,text width=0.25cm,fill=white]
    \tikzstyle{initial}=[circle,thick,draw=blue,text width=0.25cm,fill=white]
    \tikzstyle{final}=[circle,double,draw=blue,text width=0.25cm,fill=white]    
    \begin{tikzpicture}[->,>=stealth',shorten >=1pt,auto,node distance=3cm,thick]
      \node[initial] (q0) at (0,0) {0};
      \node[state] (q1) at (2,1) {1};
      \node[state] (q2) at (2,-1) {2};
      \node[state] (q3) at (4,0) {3};
      \node[state] (q4) at (6,0) {4};
      \node[final] (q5) at (8,0) {5};
      \node[final] (q6) at (8,-1) {6};
      \path[every node/.style={font=\sffamily\small,
  	  fill=white,inner sep=1pt}]
      (q0) edge [out=60,in=135] node {The/$0.3$} (q1)
      edge [out=30,in=-135] node {A/$0.2$} (q1)
      edge [out=-45,in=180] node {A/$0.3$} (q2)
      edge [out=-135,in=-135] node {This/$0.2$} (q2)
      (q1) edge [out=0,in=135] node {dog/$1$} (q3)
      (q2) edge [out=45,in=180] node {dog/$0.3$} (q3)
      edge [out=-45,in=-90] node {cat/$0.7$} (q3)
      (q3) edge node {is/$1$} (q4)
      (q4) edge [out=120,in=60,looseness=4] node {very/$0.2$} (q4)
      edge [out=0,in=180] node {cute/$0.4$} (q5)
      edge [out=-45,in=180] node {hungry/$0.4$} (q6);
    \end{tikzpicture}
  \end{center}
\end{frame}

\begin{frame}
  \frametitle{How to Determinize a WFSA}

  The only general algorithm for {\bf determinizing} a WFSA is the
  following exponential-time algorithm:
  \begin{itemize}
  \item For every state in $A$, for every set of edges
    $e_1,\ldots,e_K$ that all have the same label:
    \begin{itemize}
    \item Create a new edge, $e$, with weight $w[e]=w[e_1]\oplus\cdots\oplus w[e_K]$.
    \item Create a brand new successor state $n[e]$.
    \item For every edge leaving any of the original successor states
      $n[e_k],~1\le k\le K$, whose label is unique:
      \begin{itemize}
      \item Copy  it to $n[e]$, $\otimes$ its weight by $w[e_k]/w[e]$
      \end{itemize}
    \item For every set of edges leaving $n[e_k]$ that all have the
      same label:
      \begin{itemize}
      \item Recurse!
      \end{itemize}
    \end{itemize}
  \end{itemize}
\end{frame}

%%%%%%%%%%%%%%%%%%%%%%%%%%%%%%%%%%%%%%%%%%%%
\section[Semirings]{Semirings}
\setcounter{subsection}{1}

\begin{frame}
  \frametitle{Semirings}

  A {\bf semiring} is a set of numbers, over which it's possible to
  define a operators $\otimes$ and $\oplus$,
  and identity elements $\bar{1}$ and $\bar{0}$.
  \begin{itemize}
  \item The {\bf Probability Semiring} is the set of non-negative real
    numbers $\mathbb{R}_+$, with $\otimes=\cdot$, $\oplus=+$, $\bar{1}=1$, and $\bar{0}=0$.
  \item The {\bf Log Semiring} is the extended reals
    $\mathbb{R}\cup\left\{\infty\right\}$, with $\otimes=+$, $\oplus
    =-\logsumexp(-,-)$, $\bar{1}=0$, and $\bar{0}=\infty$.
  \item The {\bf Tropical Semiring} is just the log semiring, but with
    $\oplus=\min$.  In other words, instead of adding the
    probabilities of two paths, we choose the best path:
    \begin{displaymath}
      a\oplus b = \min(a,b)
    \end{displaymath}
  \end{itemize}
  Mohri et al. (2001) formalize it like this: a {\bf semiring} is
  $K=\left\{\mathbb{K},\oplus,\otimes,\bar{0},\bar{1}\right\}$ where
  $\mathbb{K}$ is a set of numbers.
\end{frame}

%%%%%%%%%%%%%%%%%%%%%%%%%%%%%%%%%%%%%%%%%%%%
\section[WFSTs]{How to Handle HMMs: The Weighted Finite State Transducer}
\setcounter{subsection}{1}

\begin{frame}
  \frametitle{Weighted Finite State Transducers}
  \centerline{
    \tikzstyle{state}=[circle,thin,draw=blue,text width=0.25cm,fill=white]
    \tikzstyle{initial}=[circle,thick,draw=blue,text width=0.25cm,fill=white]
    \tikzstyle{final}=[circle,double,draw=blue,text width=0.25cm,fill=white]    
    \begin{tikzpicture}[->,>=stealth',shorten >=1pt,auto,node distance=3cm,thick]
      \node[initial] (q0) at (0,0) {0};
      \node[state] (q1) at (2.5,1) {1};
      \node[state] (q2) at (2.5,-1) {2};
      \node[state] (q3) at (5,0) {3};
      \node[state] (q4) at (7.5,0) {4};
      \node[state] (q7) at (7.5,-1) {7};
      \node[final] (q5) at (10,0) {5};
      \node[final] (q6) at (10,-1) {6};
      \path[every node/.style={font=\sffamily\small,
  	  fill=white,inner sep=1pt}]
      (q0) edge [out=60,in=135] node {The:Le/$0.3$} (q1)
      edge [out=30,in=-135] node {A:Un/$0.2$} (q1)
      edge [out=-45,in=180] node {A:Un/$0.3$} (q2)
      edge [out=-135,in=-135,looseness=2] node {This:Ce/$0.2$} (q2)
      (q1) edge [out=0,in=135] node {dog:chien/$1$} (q3)
      (q2) edge [out=45,in=180] node {dog:chien/$0.3$} (q3)
      edge [out=-45,in=-90,looseness=2] node {cat:chat/$0.7$} (q3)
      (q3) edge [out=0,in=180] node {is:est/$0.5$} (q4)
      (q3) edge  [out=-45,in=135] node {is:a/$0.5$} (q7)
      (q4) edge [out=120,in=60,looseness=9] node {very:tr{\`{e}}s/$0.2$} (q4)
      edge[out=0,in=180] node {cute:mignon/$0.8$} (q5)
      (q7)  edge [out=-120,in=-60,looseness=7] node {very:tr{\`{e}}s/$0.2$} (q7)
      edge[out=0,in=180] node {hungry:faim/$0.8$} (q6);
    \end{tikzpicture}
  }
  
  A {\bf (Weighted) Finite State Transducer (WFST)} is a (W)FSA with two
  labels on every edge:
  \begin{itemize}
  \item An input label, $i\in\Sigma$, and
  \item An output label, $o\in\Omega$.
  \end{itemize}
\end{frame}

\begin{frame}
  \frametitle{What it's for}
  \begin{itemize}
  \item An {\bf FST} specifies a mapping between two sets of strings.
    \begin{itemize}
    \item The input  set is ${\mathcal I}\subset\Sigma^*$, where $\Sigma^*$ is the set of
      all strings containing zero or more letters from the alphabet $\Sigma$.
    \item The output set is ${\mathcal O}\subset\Omega^*$.
    \item For every $\vec{i}=[i_1,\ldots,i_T]\in{\mathcal I}$, the FST specifies one or more
      possible translations $\vec{o}=[o_1,\ldots,o_T]\in{\mathcal O}$.
    \end{itemize}
  \item A {\bf WFST} also specifies a probability mass function over
    the translations.  The example on the previous slide was
    normalized to compute a joint pmf $p(\vec{i},\vec{o})$, but other
    WFSAs might be normalized to compute a conditional pmf
    $p(\vec{o}|\vec{i})$, or something else.
  \end{itemize}
\end{frame}

\begin{frame}
  \frametitle{Normalizing for Conditional Probability}
  Here is a WFST whose weights are normalized to compute $p(\vec{o}|\vec{i})$:
  \centerline{
    \tikzstyle{state}=[circle,thin,draw=blue,text width=0.25cm,fill=white]
    \tikzstyle{initial}=[circle,thick,draw=blue,text width=0.25cm,fill=white]
    \tikzstyle{final}=[circle,double,draw=blue,text width=0.25cm,fill=white]    
    \begin{tikzpicture}[->,>=stealth',shorten >=1pt,auto,node distance=3cm,thick]
      \node[initial] (q0) at (0,0) {0};
      \node[state] (q1) at (2.5,1) {1};
      \node[state] (q2) at (2,-1) {2};
      \node[state] (q3) at (5,0) {3};
      \node[state] (q4) at (7.5,0) {4};
      \node[state] (q7) at (7.5,-1) {7};
      \node[final] (q5) at (10,0) {5};
      \node[final] (q6) at (10,-1) {6};
      \path[every node/.style={font=\sffamily\small,
  	  fill=white,inner sep=1pt}]
      (q0) edge [out=60,in=135] node {The:Le/$1$} (q1)
      edge [out=30,in=-135] node {A:Un/$1$} (q1)
      edge [out=-45,in=180] node {A:Un/$1$} (q2)
      edge [out=-135,in=-135,looseness=2] node {This:Ce/$1$} (q2)
      (q1) edge [out=0,in=135] node {dog:chien/$1$} (q3)
      (q2) edge [out=45,in=180] node {dog:chien/$1$} (q3)
      edge [out=0,in=-135] node {cat:f{\'{e}}lin/$0.1$} (q3)
      edge [out=-45,in=-90,looseness=2] node {cat:chat/$0.9$} (q3)
      (q3) edge [out=0,in=180] node {is:est/$0.5$} (q4)
      (q3) edge  [out=-45,in=135] node {is:a/$0.5$} (q7)
      (q4) edge [out=120,in=60,looseness=9] node {very:tr{\`{e}}s/$1$} (q4)
      edge[out=0,in=180] node {cute:mignon/$1$} (q5)
      (q7)  edge [out=-120,in=-60,looseness=7] node {very:tr{\`{e}}s/$1$} (q7)
      edge[out=0,in=180] node {hungry:faim/$1$} (q6);
    \end{tikzpicture}
  }
\end{frame}

\begin{frame}
  \frametitle{Normalizing for Conditional Probability}

  Normalizing for {\bf conditional probability} allows us to
  separately represent the two parts of a hidden Markov model.
  \begin{enumerate}
  \item The transition probabilities, $a_{ij}$, are the weights on a
    WFSA.
  \item The observation probabilities, $b_j(\vec{x}_t)$, are the
    weights on a WFST.
  \end{enumerate}
\end{frame}

\begin{frame}
  \frametitle{WFSA: Symbols on the edges are called PDFIDs}

  It is no longer useful to say that ``the labels on the edges are
  the state numbers.''  Instead, let's call them {\bf pdfids}.

  \begin{center}
    \tikzstyle{state}=[circle,thin,draw=blue,text width=0.25cm,fill=white]
    \tikzstyle{initial}=[circle,thick,draw=blue,text width=0.25cm,fill=white]
    \tikzstyle{final}=[circle,double,draw=blue,text width=0.25cm,fill=white]    
    \begin{tikzpicture}[->,>=stealth',shorten >=1pt,auto,node distance=3cm,thick]
      \node[initial] (q1) at (0,0) {1};
      \node[state] (q2) at (3,0) {2};
      \node[final] (q3) at (6,0) {3};
      \path[every node/.style={font=\sffamily\small,
  	  fill=white,inner sep=1pt}]
      (q1) edge [out=-120,in=120,looseness=4] node {1/$a_{11}$} (q1)
      edge [out=15,in=165] node {1/$a_{12}$} (q2)
      edge [out=45,in=135] node {1/$a_{13}$} (q3)
      (q2) edge [out=120,in=60,looseness=4] node {2/$a_{22}$} (q2)
      edge [out=-165,in=-15] node {2/$a_{21}$} (q1)
      edge [out=15,in=165] node {2/$a_{23}$} (q3)
      (q3) edge [out=60,in=-60,looseness=4] node {3/$a_{33}$} (q3)
      edge [out=-165,in=-15] node {3/$a_{32}$} (q2)
      edge [out=-135,in=-45] node {3/$a_{31}$} (q1);
    \end{tikzpicture}
  \end{center}
\end{frame}
  
\begin{frame}
  \frametitle{Observation Probabilities as Conditional Edge Weights}

  Now we can create a new WFST whose {\bf output symbols are pdfids} $j$, whose
  {\bf input symbols are observations}, $\vec{x}_t$, and whose {\bf weights are the
    observation probabilities,} $b_j(\vec{x}_t)$.
  \begin{center}
    \tikzstyle{state}=[circle,thin,draw=blue,text width=0.25cm,fill=white]
    \tikzstyle{initial}=[circle,thick,draw=blue,text width=0.25cm,fill=white]
    \tikzstyle{final}=[circle,double,draw=blue,text width=0.25cm,fill=white]    
    \begin{tikzpicture}[->,>=stealth',shorten >=1pt,auto,node distance=3cm,thick]
      \node[initial] (q0) at (0,0) {0};
      \node[state] (q1) at (2.5,0) {1};
      \node[state] (q2) at (5,0) {2};
      \node[state] (q3) at (7.5,0) {3};
      \node[final] (q4) at (10,0) {4};
      \path[every node/.style={font=\sffamily\small,
  	  fill=white,inner sep=1pt}]
      (q0) edge [out=45,in=135] node {\tiny{$\vec{x}_1$:1/$b_1(\vec{x}_1)$}} (q1)
      edge [out=0,in=180] node {\tiny{$\vec{x}_1$:2/$b_2(\vec{x}_1)$}} (q1)
      edge [out=-45,in=-135] node {\tiny{$\vec{x}_1$:3/$b_3(\vec{x}_1)$}} (q1)
      (q1) edge [out=45,in=135] node {\tiny{$\vec{x}_2$:1/$b_1(\vec{x}_2)$}} (q2)
      edge [out=0,in=180] node {\tiny{$\vec{x}_2$:2/$b_2(\vec{x}_2)$}} (q2)
      edge [out=-45,in=-135] node {\tiny{$\vec{x}_2$:3/$b_3(\vec{x}_2)$}} (q2)
      (q2) edge [out=45,in=135] node {\tiny{$\vec{x}_3$:1/$b_1(\vec{x}_3)$}} (q3)
      edge [out=0,in=180] node {\tiny{$\vec{x}_3$:2/$b_2(\vec{x}_3)$}} (q3)
      edge [out=-45,in=-135] node {\tiny{$\vec{x}_3$:3/$b_3(\vec{x}_3)$}} (q3)
      (q3) edge [out=45,in=135] node {\tiny{$\vec{x}_4$:1/$b_1(\vec{x}_4)$}} (q4)
      edge [out=0,in=180] node {\tiny{$\vec{x}_4$:2/$b_2(\vec{x}_4)$}} (q4)
      edge [out=-45,in=-135] node {\tiny{$\vec{x}_4$:3/$b_3(\vec{x}_4)$}} (q4);
    \end{tikzpicture}
  \end{center}
\end{frame}

\begin{frame}
  \frametitle{Hooray!  We've almost re-created the HMM!}

  So far we have:
  \begin{itemize}
  \item You can create a WFSA whose weights are the transition
    probabilities.
  \item You can create a WFST whose weights are the observation
    probabilities.
  \end{itemize}
  Here are the problems:
  \begin{enumerate}
  \item How can we combine them?
  \item Even if we could combine them, can this do anything that an HMM couldn't already do?
  \end{enumerate}
\end{frame}

%%%%%%%%%%%%%%%%%%%%%%%%%%%%%%%%%%%%%%%%%%%%
\section[Composition]{Composition}
\setcounter{subsection}{1}

\begin{frame}
  \frametitle{Composition}

  The main reason to use WFSTs is an operator called ``composition.''
  Suppose you have
  \begin{enumerate}
  \item A WFST, $R$, that translates strings $a\in{\mathcal A}$ into
    strings $b\in{\mathcal B}$ with joint probability $p(a,b)$.
  \item Another WFST, $S$, that translates strings $b\in{\mathcal B}$
    into strings $c\in{\mathcal C}$ with conditional probability $p(c|b)$.
  \end{enumerate}
  The operation $T=R\circ S$ gives you a WFST, $T$, that translates
  strings $a\in{\mathcal A}$ into strings $c\in{\mathcal C}$ with
  joint probability
  \begin{displaymath}
    p(a,c) = \sum_{b\in{\mathcal B}} p(a,b)p(c|b)
  \end{displaymath}
\end{frame}

\begin{frame}
  \frametitle{The WFST Composition Algorithm}

  \begin{enumerate}
  \item {\bf Initialize:} The initial state of $T$ is a pair,
    $i_T=(i_R,i_S)$, encoding the initial states of both $R$ and $S$.
  \item {\bf Iterate:} While there is any state $q_T=(q_R,q_S)$ with
    edges $(e_R=a:b,e_S=b:c)$ that have not yet been copied to $e_T$,
    \begin{enumerate}
    \item Create a new edge $e_T$ with next state $n[e_T]=(n[e_R],n[e_S])$
      and labels $i[e_T]:o[e_T]=i[e_R]:o[e_S] =a:c$.
    \item If an edge with the same $n[e_T]$, $i[e_T]$, and $o[e_T]$ already exists, then update
      its weight:
      \[
      w[e_T] = w[e_T]\oplus (w[e_R]\otimes w[e_S])
      \]
    \item If not, create a new edge with
      \[
      w[e_T] = w[e_R]\otimes w[e_S]
      \]
    \end{enumerate}
  \item {\bf Terminate:} A state $q_T=(q_R,q_S)$ is a final state
    if both $q_R$ and $q_S$ are final states.
  \end{enumerate}
\end{frame}

\begin{frame}
  \frametitle{Composition Example: HMM}
  \begin{tcolorbox}
  \begin{center}
    \tikzstyle{state}=[circle,thin,draw=blue,text width=0.25cm,fill=white]
    \tikzstyle{initial}=[circle,thick,draw=blue,text width=0.25cm,fill=white]
    \tikzstyle{final}=[circle,double,draw=blue,text width=0.25cm,fill=white]    
    \begin{tikzpicture}[scale=0.9,->,>=stealth',shorten >=1pt,auto,node distance=3cm,thick]
      \node[initial] (q0) at (0,0) {0};
      \node[state] (q1) at (2.5,0) {1};
      \node[state] (q2) at (5,0) {2};
      \node[state] (q3) at (7.5,0) {3};
      \node[final] (q4) at (10,0) {4};
      \path[every node/.style={font=\sffamily\small,
  	  fill=white,inner sep=1pt}]
      (q0) edge [out=45,in=135] node {\tiny{$\vec{x}_1$:1/$b_1(\vec{x}_1)$}} (q1)
      edge [out=0,in=180] node {\tiny{$\vec{x}_1$:2/$b_2(\vec{x}_1)$}} (q1)
      edge [out=-45,in=-135] node {\tiny{$\vec{x}_1$:3/$b_3(\vec{x}_1)$}} (q1)
      (q1) edge [out=45,in=135] node {\tiny{$\vec{x}_2$:1/$b_1(\vec{x}_2)$}} (q2)
      edge [out=0,in=180] node {\tiny{$\vec{x}_2$:2/$b_2(\vec{x}_2)$}} (q2)
      edge [out=-45,in=-135] node {\tiny{$\vec{x}_2$:3/$b_3(\vec{x}_2)$}} (q2)
      (q2) edge [out=45,in=135] node {\tiny{$\vec{x}_3$:1/$b_1(\vec{x}_3)$}} (q3)
      edge [out=0,in=180] node {\tiny{$\vec{x}_3$:2/$b_2(\vec{x}_3)$}} (q3)
      edge [out=-45,in=-135] node {\tiny{$\vec{x}_3$:3/$b_3(\vec{x}_3)$}} (q3)
      (q3) edge [out=45,in=135] node {\tiny{$\vec{x}_4$:1/$b_1(\vec{x}_4)$}} (q4)
      edge [out=0,in=180] node {\tiny{$\vec{x}_4$:2/$b_2(\vec{x}_4)$}} (q4)
      edge [out=-45,in=-135] node {\tiny{$\vec{x}_4$:3/$b_3(\vec{x}_4)$}} (q4);
    \end{tikzpicture}
  \end{center}
  \end{tcolorbox}
  \begin{center}
    \tikzstyle{initial}=[circle,thick,draw=black,text width=0.25cm,fill=white]
    \begin{tikzpicture}
      \node[initial] (q1) at (0,0) {};
    \end{tikzpicture}
  \end{center}
  \begin{tcolorbox}
    \begin{center}
      \tikzstyle{state}=[circle,thin,draw=blue,text width=0.25cm,fill=white]
      \tikzstyle{initial}=[circle,thick,draw=blue,text width=0.25cm,fill=white]
      \tikzstyle{final}=[circle,double,draw=blue,text width=0.25cm,fill=white]    
      \begin{tikzpicture}[scale=0.75,->,>=stealth',shorten >=1pt,auto,node distance=3cm,thick]
        \node[initial] (q1) at (0,0) {1};
        \node[state] (q2) at (3,0) {2};
        \node[final] (q3) at (6,0) {3};
        \path[every node/.style={font=\sffamily\small,
  	    fill=white,inner sep=1pt}]
        (q1) edge [out=-120,in=120,looseness=4] node {\tiny{1/$a_{11}$}} (q1)
        edge [out=15,in=165] node {\tiny{1/$a_{12}$}} (q2)
        edge [out=45,in=135] node {\tiny{1/$a_{13}$}} (q3)
        (q2) edge [out=120,in=60,looseness=4] node {\tiny{2/$a_{22}$}} (q2)
        edge [out=-165,in=-15] node {\tiny{2/$a_{21}$}} (q1)
        edge [out=15,in=165] node {\tiny{2/$a_{23}$}} (q3)
        (q3) edge [out=60,in=-60,looseness=4] node {\tiny{3/$a_{33}$}} (q3)
        edge [out=-165,in=-15] node {\tiny{3/$a_{32}$}} (q2)
        edge [out=-135,in=-45] node {\tiny{3/$a_{31}$}} (q1);
      \end{tikzpicture}
    \end{center}
  \end{tcolorbox}
\end{frame}

\begin{frame}
  \frametitle{Composition Example: HMM}
  \centerline{
    \tikzstyle{state}=[circle,thin,draw=blue,text width=0.45cm,fill=white]
    \tikzstyle{initial}=[circle,thick,draw=blue,text width=0.45cm,fill=white]
    \tikzstyle{final}=[circle,double,draw=blue,text width=0.45cm,fill=white]    
    \begin{tikzpicture}[->,>=stealth',shorten >=1pt,auto,node distance=3cm,thick]
      \node[initial] (q0) at (0,2) {0,1};
      \node[state] (q11) at (3.5,2) {1,1};
      \node[state] (q12) at (3.5,0) {1,2};
      \node[state] (q13) at (3.5,-2) {1,3};
      \node[state] (q21) at (5,2) {2,1};
      \node[state] (q22) at (5,0) {2,2};
      \node[state] (q23) at (5,-2) {2,3};
      \node[state] (q31) at (6.5,2) {3,1};
      \node[state] (q32) at (6.5,0) {3,2};
      \node[state] (q33) at (6.5,-2) {3,3};
      \node[final] (q4) at (10,-2) {4,3};
      \path[every node/.style={font=\sffamily\small,fill=white,inner sep=1pt}]
      (q0) edge [out=0,in=180] node {{$\vec{x}_1$:1/$a_{11}b_1(\vec{x}_1)$}} (q11)
      edge [out=-45,in=180] node {{$\vec{x}_1$:1/$a_{12}b_2(\vec{x}_1)$}} (q12)
      edge [out=-90,in=180] node {{$\vec{x}_1$:1/$a_{13}b_3(\vec{x}_1)$}} (q13)
      (q11) edge (q21) edge (q22) edge (q23)
      (q12) edge (q21) edge (q22) edge (q23)
      (q13) edge (q21) edge (q22) edge (q23)
      (q21) edge (q31) edge (q32) edge (q33)
      (q22) edge (q31) edge (q32) edge (q33)
      (q23) edge (q31) edge (q32) edge (q33)
      (q31) edge [out=0,in=90] node {{$\vec{x}_4$:1/$a_{13}b_1(\vec{x}_4)$}} (q4)
      (q32) edge [out=0,in=135] node {{$\vec{x}_4$:2/$a_{23}b_2(\vec{x}_4)$}} (q4)
      (q33) edge [out=0,in=180] node {{$\vec{x}_4$:3/$a_{33}b_3(\vec{x}_4)$}} (q4);
    \end{tikzpicture}
    }
\end{frame}

%%%%%%%%%%%%%%%%%%%%%%%%%%%%%%%%%%%%%%%%%%%%
\section[Epsilon]{Doing Useful Stuff: The Epsilon Transition}
\setcounter{subsection}{1}

\begin{frame}
  \frametitle{Doing Useful Stuff: The Epsilon Transition}
  \begin{itemize}
  \item There's only one more thing you need to do useful stuff: nothing.
  \item To be more precise: we can use the label $\epsilon$
    (pronounced ``epsilon'') to mean ``nothing at all.''
  \end{itemize}
\end{frame}

\begin{frame}
  \frametitle{Example: Epsilon Transitions in the Pronlex}

  \begin{itemize}
  \item A ``pronlex'' (pronunciation lexicon) is a WFST that maps from
    phoneme strings to words.
  \item A ``phoneme string'' is a sequence of many labels.  A word is just one label.
    The extra labels in the output side of the WFST all use $\epsilon$, to mean that
    they don't generate any extra output string.
  \end{itemize}
\end{frame}

\begin{frame}
  \frametitle{Example Pronlex}
  \centerline{
    \tikzstyle{state}=[circle,thin,draw=blue,text width=0.25cm,fill=white]
    \tikzstyle{initial}=[circle,thick,draw=blue,text width=0.25cm,fill=white]
    \tikzstyle{final}=[circle,double,draw=blue,text width=0.25cm,fill=white]    
    \begin{tikzpicture}[->,>=stealth',shorten >=1pt,auto,node distance=3cm,thick]
      \node[initial] (q0) at (0,0) {};
      \node[state] (q12) at (2.5,1) {};
      \node[state] (q13) at (2.5,2) {};
      \node[state] (q14) at (2.5,3) {};
      \node[state] (q22) at (5,1) {};
      \node[state] (q23) at (5,2) {};
      \node[state] (q25) at (5,4) {};
      \node[state] (q31) at (7.5,0) {};
      \node[state] (q32) at (7.5,1) {};
      \node[state] (q33) at (7.5,2) {};
      \node[state] (q34) at (7.5,3) {};
      \node[state] (q35) at (7.5,4) {};
      \node[final] (q4) at (10,0) {};
      \path[every node/.style={font=\sffamily\small,fill=white,inner sep=1pt}]
      (q0) edge [out=0,in=180] node {\ipa{[@]}:A} (q31)
      (q0) edge [out=30,in=180] node {\ipa{[k]}:$\epsilon$} (q12)
      (q0) edge [out=60,in=180] node {\ipa{[d]}:$\epsilon$} (q13)
      (q0) edge [out=90,in=180] node {\ipa{[D]}:$\epsilon$} (q14)
      (q12) edge [out=0,in=180] node {\ipa{[\ae]}:$\epsilon$} (q22)
      (q13) edge [out=0,in=180] node {\ipa{[O]}:$\epsilon$} (q23)
      (q14) edge [out=0,in=180] node {\ipa{[@]}:The} (q34)
      (q14) edge [out=45,in=180] node {\ipa{[I]}:$\epsilon$} (q25)
      (q22) edge [out=0,in=180] node {\ipa{[t]}:cat} (q32)
      (q23) edge [out=0,in=180] node {\ipa{[g]}:dog} (q33)
      (q25) edge [out=0,in=180] node {\ipa{[s]}:This} (q35)
      (q31) edge [out=0,in=180] node {$\epsilon$:$\epsilon$} (q4)
      (q32) edge [out=0,in=160] node {$\epsilon$:$\epsilon$} (q4)
      (q33) edge [out=0,in=140] node {$\epsilon$:$\epsilon$} (q4)
      (q34) edge [out=0,in=120] node {$\epsilon$:$\epsilon$} (q4)
      (q35) edge [out=0,in=100] node {$\epsilon$:$\epsilon$} (q4)
      (q4) edge [out=-150,in=-30] node {$\epsilon$:$\epsilon$} (q0);
    \end{tikzpicture}
  }
\end{frame}


\begin{frame}
  \frametitle{Example: Speech-to-Text Translation}

  \begin{itemize}
  \item For example, suppose you have some English speech.  You'd like
    to convert it to French text.
  \item Suppose you have an English pronlex, $L$, that maps English phonemes to words.
  \item You also have a translator, $G$, that maps English words to French words.
  \item Then
    \[
    T = L\circ G
    \]
    maps from English phonemes to French words.
  \end{itemize}
\end{frame}

\begin{frame}
  \frametitle{Example: Speech-to-Text Translation}
  \centerline{
    \tikzstyle{state}=[circle,thin,draw=blue,text width=0.25cm,fill=white]
    \tikzstyle{initial}=[circle,thick,draw=blue,text width=0.25cm,fill=white]
    \tikzstyle{final}=[circle,double,draw=blue,text width=0.25cm,fill=white]    
    \tikzstyle{dot}=[circle,thick,draw=black,fill=black]    
    \begin{tikzpicture}[->,>=stealth',shorten >=1pt,auto,node distance=3cm,thick]
      \node[initial] (q0) at (0,0) {};
      \node[state] (q11) at (2.5,0) {};
      \node[state] (q13) at (2.5,3) {};
      \node[state] (q21) at (5,0) {};
      \node[state] (q22) at (5,1.5) {};
      \node[state] (q23) at (5,3) {};
      \node[state] (q24) at (5,4.5) {};
      \node[state] (q31) at (7.5,0) {};
      \node[state] (q32) at (7.5,1.5) {};
      \node[state] (q33) at (7.5,3) {};
      \node[state] (q34) at (7.5,4.5) {};
      \node[dot] (d1) at (8.5,2.25) {};
      \node[dot] (d2) at (9,2.25) {};
      \node[dot] (d3) at (9.5,2.25) {};
      \path[every node/.style={font=\sffamily\small,fill=white,inner sep=1pt}]
      (q0) edge [out=0,in=180] node {\ipa{[D]}:$\epsilon$} (q11)
      (q0) edge [out=60,in=180] node {\ipa{[@]}:Un/$0.5$} (q13)
      (q11) edge [out=0,in=180] node {\ipa{[I]}:$\epsilon$} (q21)
      (q11) edge [out=45,in=180] node {\ipa{[@]}:Le/$0.2$} (q22)
      (q13) edge [out=0,in=180] node {\ipa{[k]}:$\epsilon$} (q23)
      (q13) edge [out=45,in=180] node {\ipa{[d]}:$\epsilon$} (q24)
      (q21) edge [out=0,in=180] node {\ipa{[s]}:Ce/$0.2$} (q31)
      (q22) edge [out=45,in=180] node {\ipa{[d]}:$\epsilon$/$0.2$} (q32)
      (q23) edge [out=0,in=180] node {\ipa{[\ae]}:$\epsilon$} (q33)
      (q24) edge [out=45,in=180] node {\ipa{[O]}:$\epsilon$} (q34);
    \end{tikzpicture}
  }
\end{frame}

\begin{frame}
  \frametitle{Example: Speech-to-Text Translation}

  Suppose you have:
  \begin{itemize}
  \item Observer, $B$, maps from $\vec{x}_t$ to $j$, with weights $b_j(\vec{x}_t)$.
  \item HMM, $H$, maps from $i$ and $j$ to phonemes, with weights $a_{ij}$.
  \item Pronlex, $L$, maps from phonemes to English words.
  \item Grammar, $G$, maps from English words to French words.
  \end{itemize}
  Then the translation of audio frames into French words is given by
  \[
  B\circ H\circ L\circ G
  \]
\end{frame}

%%%%%%%%%%%%%%%%%%%%%%%%%%%%%%%%%%%%%%%%%%%%
\section[Summary]{Summary}
\setcounter{subsection}{1}

\begin{frame}
  \frametitle{Weighted Finite State Transducers}
  \centerline{
    \tikzstyle{state}=[circle,thin,draw=blue,text width=0.25cm,fill=white]
    \tikzstyle{initial}=[circle,thick,draw=blue,text width=0.25cm,fill=white]
    \tikzstyle{final}=[circle,double,draw=blue,text width=0.25cm,fill=white]    
    \begin{tikzpicture}[->,>=stealth',shorten >=1pt,auto,node distance=3cm,thick]
      \node[initial] (q0) at (0,0) {0};
      \node[state] (q1) at (2.5,1) {1};
      \node[state] (q2) at (2.5,-1) {2};
      \node[state] (q3) at (5,0) {3};
      \node[state] (q4) at (7.5,0) {4};
      \node[state] (q7) at (7.5,-1) {7};
      \node[final] (q5) at (10,0) {5};
      \node[final] (q6) at (10,-1) {6};
      \path[every node/.style={font=\sffamily\small,
  	  fill=white,inner sep=1pt}]
      (q0) edge [out=60,in=135] node {The:Le/$0.3$} (q1)
      edge [out=30,in=-135] node {A:Un/$0.2$} (q1)
      edge [out=-45,in=180] node {A:Un/$0.3$} (q2)
      edge [out=-135,in=-135,looseness=2] node {This:Ce/$0.2$} (q2)
      (q1) edge [out=0,in=135] node {dog:chien/$1$} (q3)
      (q2) edge [out=45,in=180] node {dog:chien/$0.3$} (q3)
      edge [out=-45,in=-90,looseness=2] node {cat:chat/$0.7$} (q3)
      (q3) edge [out=0,in=180] node {is:est/$0.5$} (q4)
      (q3) edge  [out=-45,in=135] node {is:a/$0.5$} (q7)
      (q4) edge [out=120,in=60,looseness=9] node {very:tr{\`{e}}s/$0.2$} (q4)
      edge[out=0,in=180] node {cute:mignon/$0.8$} (q5)
      (q7)  edge [out=-120,in=-60,looseness=7] node {very:tr{\`{e}}s/$0.2$} (q7)
      edge[out=0,in=180] node {hungry:faim/$0.8$} (q6);
    \end{tikzpicture}
  }
  
  A {\bf (Weighted) Finite State Transducer (WFST)} is a (W)FSA with two
  labels on every edge:
  \begin{itemize}
  \item An input label, $i\in\Sigma$, and
  \item An output label, $o\in\Omega$.
  \end{itemize}
\end{frame}

\begin{frame}
  \frametitle{The WFST Composition Algorithm}

  \[
  T = R\circ S
  \]
  \begin{enumerate}
  \item {\bf Initialize:} The initial state of $T$ is a pair,
    $i_T=(i_R,i_S)$, encoding the initial states of both $R$ and $S$.
  \item {\bf Iterate:} Each edge $e_T=(e_R,e_S)$:
    \begin{itemize}
    \item Starts at $p[e_T]=(p[e_R],p[e_S])$
    \item Has the edge label $i[e_R]:o[e_S]$.
    \item Ends at $n[e_T]=(n[e_R],n[e_S])$.
    \item Has the weight $w[e_T]=w[e_R]\otimes w[e_S]$, possibly
      summed ($\oplus$) over nondeterministic $(e_R,e_S)$ pairs.
    \end{itemize}
  \item {\bf Terminate:} A state $q_T=(q_R,q_S)$ is a final state
    if both $q_R$ and $q_S$ are final states.
  \end{enumerate}
\end{frame}

\end{document}

